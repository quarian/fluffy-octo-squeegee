\chapter{Background in scientific literature}
\label{chapter:background} 

Here is a look to the releavant papers and such - I'll update this section once
I get to work with bibtex.

This section focuses on scientific literature.

Article \cite{piwowar2007sharing} discusses how open data increases citation
rate.

Article \cite{savage2009empirical} tried requresting data from authors who were
obligated by the publisher to publish data, but 9/10 did not share data.

Article \cite{piwowar2011shares} discusses low and slowly growing publishing
rates even on fields where publishing would be most advantageous.

Article \cite{tenopir2011data} discussed the culture and perceptions of
research data publications, showing that the culture is not well developed
and organizations don't support  researchers in their long term data storing
needs.

Article \cite{whitlock2011data} discusses best practices and such and so.
Gotta read more carefully when I'm on Aalto network.

Article \cite{wicherts2006poor} shows poor availability of data even though
they should be available.

Article \cite{alsheikh2011public} shows that even high impact papers (define
high impact later) subject to data publishing requirements don't necessarily
publish stuff. Those not subjected to any policy don't publish anything.

Article \cite{piwowar2011data} promotes good ROI on publishing research data.

Article \cite{hrynaszkiewicz2010preparing} shows some guidelines for sharing
data and stuff.

Article \cite{DBLP:journals/jasis/Borgman12} tackles the conundrum of research
data.

Article \cite{DBLP:conf/isiwi/AlamMS15} discusses an easy to use platform for
sharing data.

Article \cite{DBLP:conf/jcdl/SimonGSG15} describes a video sharing tool for research
use.

Article \cite{DBLP:journals/dlib/BermanWW14} describes RDA (research data
alliance), as a entity that promotes research data sharing.

Article \cite{DBLP:journals/jam/NohCJ14a} describes a method to encrypt private
information on a public platform.

Article \cite{DBLP:conf/ACMdis/CurmiFW14a} sharing data in social media
(biometric data), read more carefully.

Article \cite{DBLP:conf/esws/EkaputraSSB14} describes and ontology based
system for sharing research data.

Article \cite{DBLP:journals/ijdc/DoornDH13} discusses whether open publishing
of data could prevent scientific fraud following a hude fraud incident.

Article \cite{DBLP:journals/ijdc/GrootveldE12} pilots the idea of a research
data peer review.

Article \cite{wicherts2011willingness} examines if not publishing research data
means that results are in fact weaker.

Book \cite{DBLP:series/synthesis/2010Rajasekar} is the iRODS primer, cite
on.

Data intensive science (the fourth paradigm), \cite{DBLP:books/ms/4paradigm09},
things to cite regarding the data science.

CKAN platform investigation, \cite{winn2013open}.

Article \cite{DBLP:journals/fini/PolineBGGHHHHKMPSAK12} talks about sharing
data in the field of brain imaging (also tackles general issued in the sharing
field).

Article \cite{knoppers2011towards} outlines code of conduct for international
data sharing in the context of genomics.

Article \cite{cragin2010data} tackles issues in insitutional and other small
institutions.

Article \cite{DBLP:journals/fgcs/RoureGS09} is about sharing scientific
workflows.

Article \cite{kaye2012tension} is about privacy issues and how to share
healthcare data in good fashion.

Classic free lunch is over (\cite{sutter2005free}), need for concurrency
increases and at the same time computing power becomes more expensive and
storage cheaper.

Article \cite{DBLP:conf/cloudcom/DemchenkoZGWL12} discusses tha challenges
of big data infrastructure in the scientific research facilities.

Article \cite{hjorland2014curating} tackles the role of libraries and data
handling professionals in the electronic world.

Executable papers as a method to reproduce data, refer to something like
this \cite{DBLP:journals/procedia/GorpM11}.

From the good old days there is this book,advocating research data sharing
before it was cool \cite{fienberg1985sharing}.

Gathering data automatically, metadata drive and so on
\cite{DBLP:journals/jbi/HarrisTTPGC09}.

Institutional repository stuff, distributed environment (university related),
here \cite{DBLP:journals/libt/Witt08}.

User engagement required in order to make data curation success with
researchers, see here \cite{DBLP:conf/ercimdl/Martinez-UribeM09}.

Overview about who shares, what, how and so on, see
\cite{borgman2010research}.

Role of libraries in emerging e-science (libraries should curate data, they are
data management experts), \cite{heidorn2011emerging}.

Good and bad things (perceptions too) about sharing research data. Medical
field, South Africa \cite{denny2015developing}.

Time is right for repositories and sharing data (data is growing, how are you
going to handle it?), see \cite{lynch2008big}.

Article \cite{eysenbach2006citation} talks about the citation advantage you get
from publishing in an open access way.

Article \cite{DBLP:journals/oir/XiaN12} talks more about getting more citations
by publishing open access style.

Article \cite{antelman2004open} is more praise towards open publishing and
research impact.

Article \cite{lawrence2001free} is an oldie, but has a lot of citations and
does talk about the merits of free online access.

Article \cite{DBLP:journals/joi/CraigPMPA07} talks about the bias of open
access - maybe it does not actually get you anywhere. Important to bring out
opposing views - it's not all rosy in the world of open access.

Article \cite{davis2008open} is another opposing voice towards the gains of
open publishing, stating that you might reach a wider audience with open
publishing but increased citation rates might be an artefact from other
sources.

Adam article \cite{DBLP:conf/sigmod/NothaftMDZLYKAH15} would be interesing from
both big data and automatic data collection. All data is data intensive
science, so making a platform to handle those aspects is interesting.

Article \cite{cimino2010clinical} speaks about a data repository implemented
by the US National Insitutes of Health where you can get health data.

Article \cite{bax2006development} presents a free meta analysis tool for health
sciences. Possibly interesting from the point of view of analyzing data
collected from multiple sources.

Article \cite{DBLP:conf/bcb/PiredduLSZ14} shows a data oriented workflow
system. Big data, genomics analysis and workflows.

As an example of repository within a field (genomics, this time), this paper
\cite{craigon2004nascarrays} shows a system to store and share data related
to this field. Another similar system here \cite{DBLP:journals/nar/EdgarDL03}.

Collaboration and sharing is presented in paper \cite{craigon2004nascarrays} in
the context of distributed open source development.

Some words about an institutional repository \cite{gibbons2009benefits}, also
quite cleanly summarizes the considerations that come in when thinking of
deploying an institutional repository.
