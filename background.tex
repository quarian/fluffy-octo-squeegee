\chapter{Background in scientific literature}
\label{chapter:background} 

Here is a look to the releavant papers and such - I'll update this section once
I get to work with bibtex.

This section focuses on scientific literature.

\section{Little background about of open access publishing}
\label{sec:benefits_open_publishing}

Open access refers to the online literature, research data and papers in the
context of this thesis, that are free of charge and openly available for reuse
and redistribution (citation). Research papers have had a longer history in
publishing, being the de facto way of publishing results and making research
visible and guidelines for open access related to papers exist (citation).

Open access publishing has been studied a lot, ranging from different
implementations (citation) to the effects of the Internet being a catalyst
for making open publishing posisble in a cost efficient large scale manner
(citation). Many nstitutions, such as universities, governments and other
funding bodies are advocating open access publishing (citation). Open access
is generally thought to increase the citation rate of a paper, making research
more impactful and since many studies are publicly funded, the perception is
that the results should be publicly funded as well.

This thesis focuses on publishing research data.

\section{The challenges of research data to open access}
\label{sec:research_data_oa}

Unlike reserch papers, research data is lacking in good guidelines and
practices to publish data in an open access way (citation). There is a
multitude of reasons behind that. The next sections (forward reference) will
describe those.

\subsection{Research data is diverse}

While research papers generally follow a
strict structure and can easily be published in a PDF format, research data
comes in all forms and flavors. For example, research data from a brain imaging
study will look completely different from a dataset that originates from
sociological research (citation). Not only the shape of the data is different,
the size of datasets can vary from the few megabyte survey results to the
multiple petabyte simulation results from a particle collider.

Even when you get down to practicalities the diversity of research data offers
challenges - results generated from one software might be binary incompatible
with another software within one field of science. And even if you were using
the same software, our versions might be different or you might do things in
such a different way that you are not compatible with other working on the
same field.

\subsection{Research data requires metadata}

Metadata refers to data about data. It's a rescription of the data and entails
details such as when was the dataset collected, what kind of equipement was
used or if there was something noteworthy in how the experiment was run
(citation). Due to the diversity of data in different fields, the metadata
differs between fields as well (citation). Some standards for metadata exist,
but not an all fields of science (citaiton).

Research data in and on itself is the most important thing to be published, but
working on someone else's research data without the relevant metadata adds
another layer of complexity and might mean that you cannot use the data at all
(citation). One reason to publish your data is to make reproduciton of you work
(citation) easier, and without proper metadata it is hard or impossible to
do that.

\subsection{Research data might be a privacy issue}

In many fields of science (healthcare, telecommunication to name a few) the
privacy of the participants to the studies is essential. If you go about
publishing your data, you need to be concious about maintaning the anonymity
of the people taking part in your research (citation).

\subsection{Research data might be published locally}

Due to privacty issues or maybe funding issues it might be so that the data
that is used in research cannot be published to the world but maybe you can
publish it within your orgnanization or upon request. This sets access rights
requirements to your dataset, so while you might be able to openly publish
your metadata in order to tell the world that your dataset exists, your dataset
should not be available for everyone (citation). 

\subsection{Research data requires support from higher ups and related partners}

Individuals can publish their data any way they wish as long as it's within
legal boundaries. However, if institutions such as universities or librareis
wish to venture into publishing their research, individual researchers and
people working with the data need support from their organizations. Studies
have shown (citation) that organizations need to commit to the idea of open
access publishing in order for it to work.

\section{Benefits of publishing research data}

The benefits of research data publishing are not as straightforward as
publishing papers. Academic credit is distributed by the number of papers you
publish and by the number of citations you get to those papers.

\section{General points}

Points to touch on this section:

\begin{itemize}
    \item sharing data and open publishing of results has been studied
    \item generally it's noted that sharing papers online leads to benefits
    \begin{itemize}
        \item citation rate increases
        \item impact of your research grows
        \item you reach wider audience
    \end{itemize}
    \item some research contradicts the increase in citation (comparing open
          access and closed citation), claiming that the increased citation
          rate of open access things is a statistical artefact
    \item in addition, there are papers on how you should organize your data
          sharing ways
    \item funding bodies have started demanding the publication of data
    \begin{itemize}
        \item however, people who have published in papers that require you
              to publish data don't actually do that
        \item publishing data is rare even on fields where sharing would be
              great
    \end{itemize}
    \item privacy and security is a concern
    \item present some systems available
    \item some technologies and platforms could be mentioned
    \begin{itemize}
        \item iRODS
        \item CKAN
        \item genomics sharing platforms
    \end{itemize}
    \item and then there are fringe cases related
    \begin{itemize}
        \item Adam
        \item Galaxy + Hadoop integration
    \end{itemize}
    \item notes about big data (paper about the difficulties of big data in
          research infrastructure
\end{itemize}

Article \cite{piwowar2007sharing} discusses how open data increases citation
rate.

Article \cite{savage2009empirical} tried requresting data from authors who were
obligated by the publisher to publish data, but 9/10 did not share data.

Article \cite{piwowar2011shares} discusses low and slowly growing publishing
rates even on fields where publishing would be most advantageous.

Article \cite{tenopir2011data} discussed the culture and perceptions of
research data publications, showing that the culture is not well developed
and organizations don't support  researchers in their long term data storing
needs.

Article \cite{whitlock2011data} discusses best practices and such and so.
Gotta read more carefully when I'm on Aalto network.

Article \cite{wicherts2006poor} shows poor availability of data even though
they should be available.

Article \cite{alsheikh2011public} shows that even high impact papers (define
high impact later) subject to data publishing requirements don't necessarily
publish stuff. Those not subjected to any policy don't publish anything.

Article \cite{piwowar2011data} promotes good ROI on publishing research data.

Article \cite{hrynaszkiewicz2010preparing} shows some guidelines for sharing
data and stuff.

Article \cite{DBLP:journals/jasis/Borgman12} tackles the conundrum of research
data.

Article \cite{DBLP:conf/isiwi/AlamMS15} discusses an easy to use platform for
sharing data.

Article \cite{DBLP:conf/jcdl/SimonGSG15} describes a video sharing tool for research
use.

Article \cite{DBLP:journals/dlib/BermanWW14} describes RDA (research data
alliance), as a entity that promotes research data sharing.

Article \cite{DBLP:journals/jam/NohCJ14a} describes a method to encrypt private
information on a public platform.

Article \cite{DBLP:conf/ACMdis/CurmiFW14a} sharing data in social media
(biometric data), read more carefully.

Article \cite{DBLP:conf/esws/EkaputraSSB14} describes and ontology based
system for sharing research data.

Article \cite{DBLP:journals/ijdc/DoornDH13} discusses whether open publishing
of data could prevent scientific fraud following a hude fraud incident.

Article \cite{DBLP:journals/ijdc/GrootveldE12} pilots the idea of a research
data peer review.

Article \cite{wicherts2011willingness} examines if not publishing research data
means that results are in fact weaker.

Book \cite{DBLP:series/synthesis/2010Rajasekar} is the iRODS primer, cite
on.

Data intensive science (the fourth paradigm), \cite{DBLP:books/ms/4paradigm09},
things to cite regarding the data science.

CKAN platform investigation, \cite{winn2013open}.

Article \cite{DBLP:journals/fini/PolineBGGHHHHKMPSAK12} talks about sharing
data in the field of brain imaging (also tackles general issued in the sharing
field).

Article \cite{knoppers2011towards} outlines code of conduct for international
data sharing in the context of genomics.

Article \cite{cragin2010data} tackles issues in insitutional and other small
institutions.

Article \cite{DBLP:journals/fgcs/RoureGS09} is about sharing scientific
workflows.

Article \cite{kaye2012tension} is about privacy issues and how to share
healthcare data in good fashion.

Classic free lunch is over (\cite{sutter2005free}), need for concurrency
increases and at the same time computing power becomes more expensive and
storage cheaper.

Article \cite{DBLP:conf/cloudcom/DemchenkoZGWL12} discusses tha challenges
of big data infrastructure in the scientific research facilities.

Article \cite{hjorland2014curating} tackles the role of libraries and data
handling professionals in the electronic world.

Executable papers as a method to reproduce data, refer to something like
this \cite{DBLP:journals/procedia/GorpM11}.

From the good old days there is this book,advocating research data sharing
before it was cool \cite{fienberg1985sharing}.

Gathering data automatically, metadata drive and so on
\cite{DBLP:journals/jbi/HarrisTTPGC09}.

Institutional repository stuff, distributed environment (university related),
here \cite{DBLP:journals/libt/Witt08}.

User engagement required in order to make data curation success with
researchers, see here \cite{DBLP:conf/ercimdl/Martinez-UribeM09}.

Overview about who shares, what, how and so on, see
\cite{borgman2010research}.

Role of libraries in emerging e-science (libraries should curate data, they are
data management experts), \cite{heidorn2011emerging}.

Good and bad things (perceptions too) about sharing research data. Medical
field, South Africa \cite{denny2015developing}.

Time is right for repositories and sharing data (data is growing, how are you
going to handle it?), see \cite{lynch2008big}.

Article \cite{eysenbach2006citation} talks about the citation advantage you get
from publishing in an open access way.

Article \cite{DBLP:journals/oir/XiaN12} talks more about getting more citations
by publishing open access style.

Article \cite{antelman2004open} is more praise towards open publishing and
research impact.

Article \cite{lawrence2001free} is an oldie, but has a lot of citations and
does talk about the merits of free online access.

Article \cite{DBLP:journals/joi/CraigPMPA07} talks about the bias of open
access - maybe it does not actually get you anywhere. Important to bring out
opposing views - it's not all rosy in the world of open access.

Article \cite{davis2008open} is another opposing voice towards the gains of
open publishing, stating that you might reach a wider audience with open
publishing but increased citation rates might be an artefact from other
sources.

Adam article \cite{DBLP:conf/sigmod/NothaftMDZLYKAH15} would be interesing from
both big data and automatic data collection. All data is data intensive
science, so making a platform to handle those aspects is interesting.

Article \cite{cimino2010clinical} speaks about a data repository implemented
by the US National Insitutes of Health where you can get health data.

Article \cite{bax2006development} presents a free meta analysis tool for health
sciences. Possibly interesting from the point of view of analyzing data
collected from multiple sources.

Article \cite{DBLP:conf/bcb/PiredduLSZ14} shows a data oriented workflow
system. Big data, genomics analysis and workflows.

As an example of repository within a field (genomics, this time), this paper
\cite{craigon2004nascarrays} shows a system to store and share data related
to this field. Another similar system here \cite{DBLP:journals/nar/EdgarDL03}.

Collaboration and sharing is presented in paper \cite{craigon2004nascarrays} in
the context of distributed open source development.

Some words about an institutional repository \cite{gibbons2009benefits}, also
quite cleanly summarizes the considerations that come in when thinking of
deploying an institutional repository.

Article \cite{DBLP:conf/icegov/SayogoP11} is yet another paper describing
the detriments to scientific data sharing.

Article \cite{irodsinproceedings} is the thing Jyväskylä wrote about the iRods
native cross GUI client.

Article \cite{DBLP:conf/elpub/Hedlund08} is about gouging the attitudes of
business people towards open publishing.

Article \cite{laakso2011development} talks about the evolution of open
publishing at the age of the internet, and points out that open publishing
is of course become more common and cheaper with internet.

Article \cite{suber2007open} is a short overview of open access publishing.

Article \cite{harnad2004comparing} boasts a bigger amount of citations to
papers that are openly accessible to the papers that are not (published in the
same journals even).
