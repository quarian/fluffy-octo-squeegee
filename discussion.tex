\chapter{Discussion}
\label{chapter:discussion}

\section{Methods}

In this thesis we examined the challenges around research data management,
sharing and publishing. The methods we used were literature reviews,
interviews, surveys, technical benchmarks and user tests in the form of contextual
interviews and lead user tests.

Literature reviews on the subject revealed that while the open access way of
publishing research papers has been studied, the scientific contribution to
the problem of open access research data has been tackled by a small number
of researchers. Statistical studies of open access benefits with research
papers promise good results for publishing open access style, but the
methodology and sample size quality varies. The metrics and numbers about
research data sharing and management are all different within the research
papers published, making it harder to interpret results from them. This is to
be expected, since research data publication and sharing is a relatively new
phenomenon. A challenge for the future would be to rigorously quantify the
benefits of research data publication and sharing. Many of the research papers
in the field are also quite new, so it will take time to find out which ones
of them provide to be the most valuable.

Interviews were used to find out the current situation of research data
management and sharing in Finland. The interviewees represented many groups
of stakeholders, but it is possible that the people chosen for the interviews
were not the best representatives of their group. Legal issues were also
brought up quite a few times during this thesis and it would have been
beneficial to interview a legal expert as well. The interviews were conducted
in places chosen by the interviewees to make the interview process easier
for them. One possible failure case with the interviews was that they were
conducted alone, but this was taken into account by recording interviews.

The surveys were not designed for the use of this thesis, instead they were used
to find out the research data management needs of Aalto for planning the
future of services offered by Aalto. However, the questions asked in the
surveys were very relevant for this thesis as well which led to the decision
to use them as is instead of doing overlapping work. The general caveats of
surveys, such as the shallowness of information and the chance of
misinterpretation, apply of course. To minimize the the risk of misinformation
the results from the surveys were shown as is without too much interpretation.

Technical benchmarking was done with existing solutions to varying degree.
The Dataverse solution was inspected thoroughly, Zenodo and Hydra installations,
other publishing platforms were signed up for and tried that way and iRODS
was studied through the documentation and the code. There is a possibility
that the conclusions drawn from the thorough inspection of Dataverse don't
apply for the other solutions directly, but since the implementations are quite
similar the risk is likely low. We tried separating the learnings from the
systems from the system specific features, such as the look and feel of their
UIs. The cloud environment that was used as the basis of the prototype
Dataverse system would no the installation destination of the final system,
but it did not seem to generate problems for the testing.

The contextual interviews and lead user tests were chosen as a tool to learn
more about the research data publication systems to complement the technical
examination. It seemed that surveys would not provide adequate information on
a tool that did not exist yet and since Aalto is forming the data policy to
bring this sort of tools to the users it seemed that getting user feedback
in a controlled fashion from these tools would be beneficial. The sample size
of 12 users (10 for the contextual interviews and 2 for the lead user tests)
does not yield statistically significant results, but the depth that the
examination was carried out with should provide value.

It was interesting that the interactions with the users of the system brought
up similar points that were raised in the literature. The lack of culture and
reasons for not sharing data were found out to be the same within these users.
Again, the sample size being so small this is nothing to write home about, but
it seems likely that the problems reported in literature around the world do
happen in Finland as well.

The goal of combining these methods was to gain a holistic view on the problem
and not only focus on the technological side.

The chosen methodologies were chosen also partially to accommodate the studies
of the writer of this thesis - I have studied user centered design as a my
minor during my masters' studies.

\section{Combining insights}

While the main focus of this thesis from the beginning was to find
appropriate technical solutions for sharing research data it became clear
that sharing research data is not strictly a technical problem. While there is
a lot of work to be done to make tools and systems to make the process of
sharing and publishing research data easy, a point solution to do just that
would fall flat. Firstly, research data publication and sharing is closely
tied to research data management during the research process. If research data
is not handled during the process with the goal of one day making it public,
the process of eventually making it public is very hard or impossible. And even
though it might be technically viable the time and effort it would take to
turn datasets that have not been properly documented and maintained during the
process might be prohibitive. Secondly, the culture for sharing research data
is not there yet. There is no knowledge about either research data management
or sharing. Additionally, there is no incentive to share research data, since
researchers' contribution is measured mainly in citation to research papers.

The problem of research data also involves people from different disciplines
in unprecedented ways. The roles of libraries, university software
infrastructure maintainers and the researchers themselves are undefined in
the new world of research data management, publication and sharing. And then
if you throw in the publishers and the peer review process that is in the
heart of the scientific publishing game it is clear that the roles of all
these actors need to be defined in the research data context.

Making research data accessible is likely to promote the quality of science.
To achieve this, better technical solutions need to be implemented and the
culture around research data needs to be taken into a more open direction.

\section{Future work}

In the previous section we discussed what would be need to make research data
more open. Future work around the subject should include both integrated
technical solutions that make good research data management a part of the
daily life of researchers and thus making the publication of their data easy.
The change the culture more research on the benefits of open research data
is required, but there might be other ways as well.

Some fields of science, such as physics, psychology and genomics have been
more successful than other in implementing open research data practices.
Research on how they managed that transition and how that could be applied
to other fields could find ways to bring open research data to other
fields as well.

It's also possible that the culture of research data could be changed with
appropriate sticks and carrots. If funding bodies would make it a priority
to demand public research data, there would likely be an urgency to provide
such solutions for researchers as soon as possible. On the other hand, maybe
citations to research data could be integrated to the h-index of researchers,
thus making the already used metric reward those who publish their datasets.
This reframing of the problem - instead of asking "how should we implement
such systems that help with research data?" we ask "how do we motivate people
to share their research data" - is a design paradigm that has been used
successfully in other system level design problems \cite{dorst2015frame}.

The solutions for the technical and cultural problems might also be found
from analogous problems. For example, software development has benefited for
a long time from the open source community, where people contribute code for
the public domain without compensation. It's been studied that people do not
do this out of the kindness of their hearts, but instead with the goal of
turning their free work now into profit in the future
\cite{DBLP:conf/hicss/HarsO01}. The tools that are used to share software,
such as GitHub, could also provide a fresh look on how tools for research
data management and sharing could work.

\iffalse
In this chapter we'll discuss the findings, methods and such in a good
scientific manner.

At some point I want to discuss the analogue of software open source community
and how that works - how we should make people proud of making good, usable
datasets and sharing them.

\cite{DBLP:conf/hicss/HarsO01} is about the motivations behind working on open
source software - could be used for analogue for open data.

\cite{dorst2015frame} is the Kees Dorst book about reframing problems, which in
this case would be taking the technical problem of sharing reserch data, which
is not the actual problem, and reframing it as the problem on how to motivat
people to to share their data.
\fi
