Lilla Vargens universum {\"a}r det tredje fiktiva universumet inom huvudf{\r{a}}ran av de
tecknade disneyserierna - de {\"o}vriga tv{\r{a}} {\"a}r Kalle Ankas och Musse Piggs
universum. Figurerna runt Lilla Vargen kommer huvudsakligen fr{\r{a}}n tre k{\"a}llor ---
dels persongalleriet i kortfilmen Tre sm{\r{a}} grisar fr{\r{a}}n 1933 och dess uppf{\"o}ljare,
dels l{\r{a}}ngfilmen S{\r{a}}ngen om S{\"o}dern fr{\r{a}}n 1946, och dels fr{\r{a}}n episoden ``Bongo'' i
l{\r{a}}ngfilmen Pank och f{\r{a}}gelfri fr{\r{a}}n 1947. Framf{\"o}r allt de tv{\r{a}} f{\"o}rsta har
sedermera {\"a}ven kommit att leva vidare, utvidgas och inf{\"o}rlivas i varandra genom
tecknade serier, fr{\"a}mst s{\r{a}}dana producerade av Western Publishing f{\"o}r
amerikanska Disneytidningar under {\r{a}}ren 1945--1984.

V{\"a}rlden runt Lilla Vargen {\"a}r, i j{\"a}mf{\"o}relse med den runt Kalle Anka eller Musse
Pigg, inte helt enhetlig, vilket bland annat m{\"a}rks i Bror Bj{\"o}rns skiftande
personlighet. Den har {\"a}ven varit betydligt mer {\"o}ppen f{\"o}r influenser fr{\r{a}}n andra
Disneyv{\"a}rldar, inte minst de tecknade l{\r{a}}ngfilmerna. Ytterligare en skillnad {\"a}r
att varelserna i vargserierna f{\"o}refaller st{\r{a}} n{\"a}rmare sina f{\"o}rebilder inom den
verkliga djurv{\"a}rlden. Att vargen Zeke vill {\"a}ta upp grisen Bror Duktig {\"a}r till
exempel ett st{\"a}ndigt {\r{a}}terkommande tema, men om katten Svarte Petter skulle f{\r{a}}
f{\"o}r sig att {\"a}ta upp musen Musse Pigg skulle detta antagligen h{\"o}ja ett och annat
{\"o}gonbryn bland l{\"a}sarna.
