\chapter{Prototype Solution}
\label{chapter:prototype}

In this chapter we'll explain the prototype solution. Prototype is one outcome
of the thesis, but so should be the insight generated by the user testing.

\section{Users of the system}
\label{sec:users}

In order to quantify the requirements of the system, we have boiled them down
to user stories. Here they are (and at this stage the user stories are from
Richard - I'll update them accordingly once I get around to do that).

\subsection{Researcher}

\begin{itemize}
    \item As a researcher, I want to publish a dataset as Open Data so that it
          can be useful to others and I can get citations.
    \item As a researcher, I want to have a single citable URL for a dataset
          which I can't release publicly, so that I can draw attention to my
          work and get citations. 
    \item As a researcher, I want to publish a massive dataset in a way that
          others can access it, and not have to duplicate metadata entry on the
          other hosting service and the dataset repository, so that I don't have
          to worry about hosting myself. 
    \item As a researcher or research group, I want my data to be linked to my
          own pages and possibly a data profile page, so that I gain visibility
          to myself from releasing data. 
    \item As a person applying for funding, I want a ready-made data publication
          solution (possibly including text) that I can put into my
          applications, so that it saves me time when applying for funding and
          increase my chances. 
    \item As a researcher, I want to store metadata in dataverse, but in a way
          that I can guarantee that it will never be published or viewable to
          anyone unauthorized by accident and without explicit consent.  (This
          is add-on to "private data" stories above.)
    \item Finding datasets online should be made easy and storing that information
          should be stored easily.
\end{itemize}

\subsection{Course}

\begin{itemize}
    \item As a course instructor in a data-driven course, I want to put several
          small datasets online for use of students in my class, so that I have
          less data management to do myself in the long run.

        \item As a course instructor in a data-driven course, I want for my students
              to be able to find other datasets online, so that they may find other
              data which engages them more than my own.
\end{itemize}

\subsection{Reseach group}

\begin{itemize}
    \item As a research group, we want to be able to put dataset documentation
          in one central place, so that we don't lose memory of data and
          metadata over time as people come and go.  This can be both published
          and private data.
    \item As a research group, we want to put metadata about our data where it
          can be browsed by others at Aalto, so that we can find collaborators
          who may want to work with us.  This data should not be visible to
          anyone else on the internet.
    \item As a research group, we want to have a data collection with
          permissions such that it is very easy to add new members to our
          group. Preferably, this is automatic using university LDAP.
    \item As a research group, we want different levels of privacy of data to
          coexist in our data collection so that we need only one collection.
          For example, all data in our collection should be private by default,
          but some can be published.
\end{itemize}

\subsection{Librarian}

\subsection{Student}

\subsection{Anybody}

\subsection{Other}

\begin{itemize}
    \item As a user of the prototype dataverse, I want to be able to migrate my
          (meta)data and permissions to the final Aalto data repository so that
          my work in preparing data is as useful as possible.
    \item As a administrator, I want to be able to see the data produced and
          released by my unit in the last N years, so that I can document
          productivity and better apply for funding.
    \item As the data repository administrator, I want tangible benefits to
          researchers, so that they feel that using this is worth their time and
          continue using long-term.  (This is vague - just something to keep in
          mind.)
\end{itemize}

\section{System description}
\label{sec:system_description}

Here we'll describe the system with the softwarae stack and relevant things.

At this stage there is going to be at least one prototype implemented using
Harvard Dataverse as the basis. Dataverse works quite well because it's easy
to install and has an easy way to manage users.

I'm currently setting up a Hydra head to see how that works. I tried Invenio
(the CERN originated system), but either their public github is not up to
date or there is something seriously wrong with their stuff because I did not
get it to work. Though Damine (leads the EUDAT project at CSC) did mention that
the Invenio thing was hard to install, so there's that.

\section{System testing}
\label{sec:system_testing}

Prototypes are naturally worthless without testing with users. Once we have a
system up and running, we'll gather notes and such to have things to say here.

\section{Outcomes of the prototype}
\label{sec:prototype_outcomes}
