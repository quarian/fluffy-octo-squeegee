\chapter{Users of Data Repositories}
\label{chapter:first-appendix}

\section{Research scientist}

\begin{compactitem}
    \item As a researcher, I want to publish a dataset as Open Data so that it
          can be useful to others and I can get citations.
    \item As a researcher, I want to have a single citable URL for a dataset
          which I can not release publicly, so that I can draw attention to my
          work and get citations. 
    \item As a researcher, I want to publish a massive dataset in a way that
          others can access it, and not have to duplicate metadata entry on the
          other hosting service and the dataset repository, so that I do not have
          to worry about hosting myself. 
    \item As a researcher or research group, I want my data to be linked to my
          own pages and possibly a data profile page, so that I gain visibility
          to myself from releasing data. 
    \item As a person applying for funding, I want a ready-made data publication
          solution (possibly including text) that I can put into my
          applications, so that it saves me time when applying for funding and
          increase my chances. 
    \item As a researcher, I want to store metadata in dataverse, but in a way
          that I can guarantee that it will never be published or viewable to
          anyone unauthorized by accident and without explicit consent.  (This
          is add-on to "private data" stories above.)
    \item Finding datasets online should be made easy and storing that information
          should be stored easily.
    \item As a researcher, I want to know what other researchers are doing and
          what kind of data they are using. I might want to collaborate with
          them or use their data to make my research
    \item As a researcher, I want my papers and research to have as much impact
          as possible, which is aided by publishing my datasets free online
    \item As a researcher, I want to be sure that my data is safe so that I can
          focus on more important work
    \item As a researcher, I want data management during my research project to
          be as easy as possible so that it does not cause unnecessary overhead
\end{compactitem}

\section{Course}

\begin{compactitem}
    \item As a course instructor in a data-driven course, I want to put several
          small datasets online for use of students in my class, so that I have
          less data management to do myself in the long run.
    \item As a course instructor in a data-driven course, I want for my students
          to be able to find other datasets online, so that they may find other
          data which engages them more than my own.
\end{compactitem}

\section{Reseach group}

\begin{compactitem}
    \item As a research group, we want to be able to put dataset documentation
          in one central place, so that we do not lose memory of data and
          metadata over time as people come and go.  This can be both published
          and private data.
    \item As a research group, we want to put metadata about our data where it
          can be browsed by others at Aalto, so that we can find collaborators
          who may want to work with us.  This data should not be visible to
          anyone else on the internet.
    \item As a research group, we want to have a data collection with
          permissions such that it is very easy to add new members to our
          group. Preferably, this is automatic using university LDAP.
    \item As a research group, we want different levels of privacy of data to
          coexist in our data collection so that we need only one collection.
          For example, all data in our collection should be private by default,
          but some can be published.
\end{compactitem}

\section{Librarian}

\begin{compactitem}
    \item As a librarian, I want to be a part of the digital publishing of out
          insitutions datasets.
    \item As a librarian, I want to use my metadata and description expertise
          to aid the researchers and to have their datasets properly documented
          and metadata properly used
    \item As a librarian, I want the dataset publication system to communicate
          with other electronic publishing systems already in use in our
          institution
    \item As a librarian, I want to pass on my knowledge of metadata and
          electronic publishing also to the scientists so that they can both
          take that into account while they work and so that my work in helping
          them becomes easier
\end{compactitem}

\section{Student}

\begin{compactitem}
    \item As a student, I want easy access to any materials that my course
          requires me to use with my course work
    \item As a student working on any level of thesis I would like to know if
          relevant datasets that I could use with my thesis exist and would
          like to have easy access to them
    \item As a student working on different studenr projects that generate
          data, I would like to know the best practices that allow me to use
          the least time to manage my data and allow me to save them in a
          convenient location
\end{compactitem}

\section{Anybody (not affiliated with the insitution hosting the repository)}

\begin{compactitem}
    \item As a developer of applications, I want to have access to interesting
          datasets that could potentially be used as a basis of great
          applications
    \item As a funding body, I want that the access to the to the datasets that
          should be public is easy and that the system does not allow them to
          be published in a way that they are hard to find
\end{compactitem}

\section{IT staff}

\begin{compactitem}
    \item As an administrator, I want to be able to see the data produced and
          released by my unit in the last N years, so that I can document
          productivity and better apply for funding.
    \item As the data repository administrator, I want tangible benefits to
          researchers, so that they feel that using this is worth their time and
          continue using long-term.  (This is vague - just something to keep in
          mind.)
    \item As an administrator, I want the system to be easy to maintain, so
          that it do not cause extraneous overhead with my other maintenance
          tasks
    \item As an administrator, I want the data management plan to contain the
          whole lidecycle of the data in order to preserve it the best way
          possible
\end{compactitem}

\section{Other}

\begin{compactitem}
    \item As a user of the prototype dataverse, I want to be able to migrate my
          (meta)data and permissions to the final Aalto data repository so that
          my work in preparing data is as useful as possible.
\end{compactitem}




\iffalse
This is the first appendix. You could put some test images or verbose data in an
appendix, if there is too much data to fit in the actual text nicely.

For now, the Aalto logo variants are shown in Figure~\ref{fig:aaltologo}.

\begin{figure}
\begin{center}
 \begin{subfigure}[b]{\textwidth}
  {\selectlanguage{english}\AaltoLogoSmall{1}{!}{aaltoBlue}}
  \caption{In English}
 \end{subfigure}
 \begin{subfigure}[b]{\textwidth}
  {\selectlanguage{finnish}\AaltoLogoSmall{1}{''}{aaltoRed}}
  \caption{Suomeksi}
 \end{subfigure}
 \begin{subfigure}[b]{\textwidth}
  {\selectlanguage{swedish}\AaltoLogoSmall{1}{?}{aaltoYellow}}
  \caption{P\r{a} svenska}
 \end{subfigure}
\caption{Aalto logo variants}
\label{fig:aaltologo}
\end{center}
\end{figure}
\fi

\chapter{Requirements for research data management and publishing system}
\label{chapter:reqs}
