\chapter{Users of Data Repositories}
\label{chapter:first-appendix}

\footnotesize

\section{Research scientist}

\begin{compactitem}
    \item As a researcher, I want to publish a dataset as Open Data so that it
          can be useful to others and I can get citations.
    \item As a researcher, I want to have a single citable URL for a dataset
          which I can not release publicly, so that I can draw attention to my
          work and get citations. 
    \item As a researcher, I want to publish a massive dataset in a way that
          others can access it, and not have to duplicate metadata entry on the
          other hosting service and the dataset repository, so that I do not have
          to worry about hosting myself. 
    \item As a researcher or research group, I want my data to be linked to my
          own pages and possibly a data profile page, so that I gain visibility
          to myself from releasing data. 
    \item As a person applying for funding, I want a ready-made data publication
          solution (possibly including text) that I can put into my
          applications, so that it saves me time when applying for funding and
          increase my chances. 
    \item As a researcher, I want to store metadata in dataverse, but in a way
          that I can guarantee that it will never be published or viewable to
          anyone unauthorized by accident and without explicit consent.  (This
          is add-on to "private data" stories above.)
    \item Finding datasets online should be made easy and storing that information
          should be stored easily.
    \item As a researcher, I want to know what other researchers are doing and
          what kind of data they are using. I might want to collaborate with
          them or use their data to make my research
    \item As a researcher, I want my papers and research to have as much impact
          as possible, which is aided by publishing my datasets free online
    \item As a researcher, I want to be sure that my data is safe so that I can
          focus on more important work
    \item As a researcher, I want data management during my research project to
          be as easy as possible so that it does not cause unnecessary overhead
\end{compactitem}

\section{Course}

\begin{compactitem}
    \item As a course instructor in a data-driven course, I want to put several
          small datasets online for use of students in my class, so that I have
          less data management to do myself in the long run.
    \item As a course instructor in a data-driven course, I want for my students
          to be able to find other datasets online, so that they may find other
          data which engages them more than my own.
\end{compactitem}

\section{Reseach group}

\begin{compactitem}
    \item As a research group, we want to be able to put dataset documentation
          in one central place, so that we do not lose memory of data and
          metadata over time as people come and go.  This can be both published
          and private data.
    \item As a research group, we want to put metadata about our data where it
          can be browsed by others at Aalto, so that we can find collaborators
          who may want to work with us.  This data should not be visible to
          anyone else on the internet.
    \item As a research group, we want to have a data collection with
          permissions such that it is very easy to add new members to our
          group. Preferably, this is automatic using university LDAP.
    \item As a research group, we want different levels of privacy of data to
          coexist in our data collection so that we need only one collection.
          For example, all data in our collection should be private by default,
          but some can be published.
\end{compactitem}

\section{Librarian}

\begin{compactitem}
    \item As a librarian, I want to be a part of the digital publishing of out
          insitutions datasets.
    \item As a librarian, I want to use my metadata and description expertise
          to aid the researchers and to have their datasets properly documented
          and metadata properly used
    \item As a librarian, I want the dataset publication system to communicate
          with other electronic publishing systems already in use in our
          institution
    \item As a librarian, I want to pass on my knowledge of metadata and
          electronic publishing also to the scientists so that they can both
          take that into account while they work and so that my work in helping
          them becomes easier
\end{compactitem}

\section{Student}

\begin{compactitem}
    \item As a student, I want easy access to any materials that my course
          requires me to use with my course work
    \item As a student working on any level of thesis I would like to know if
          relevant datasets that I could use with my thesis exist and would
          like to have easy access to them
    \item As a student working on different studenr projects that generate
          data, I would like to know the best practices that allow me to use
          the least time to manage my data and allow me to save them in a
          convenient location
\end{compactitem}

\section{Anybody (not affiliated with the insitution hosting the repository)}

\begin{compactitem}
    \item As a developer of applications, I want to have access to interesting
          datasets that could potentially be used as a basis of great
          applications
    \item As a funding body, I want that the access to the to the datasets that
          should be public is easy and that the system does not allow them to
          be published in a way that they are hard to find
\end{compactitem}

\section{IT staff}

\begin{compactitem}
    \item As an administrator, I want to be able to see the data produced and
          released by my unit in the last N years, so that I can document
          productivity and better apply for funding.
    \item As the data repository administrator, I want tangible benefits to
          researchers, so that they feel that using this is worth their time and
          continue using long-term.  (This is vague - just something to keep in
          mind.)
    \item As an administrator, I want the system to be easy to maintain, so
          that it do not cause extraneous overhead with my other maintenance
          tasks
    \item As an administrator, I want the data management plan to contain the
          whole lidecycle of the data in order to preserve it the best way
          possible
\end{compactitem}

\section{Other}

\begin{compactitem}
    \item As a user of the prototype dataverse, I want to be able to migrate my
          (meta)data and permissions to the final Aalto data repository so that
          my work in preparing data is as useful as possible.
\end{compactitem}




\iffalse
This is the first appendix. You could put some test images or verbose data in an
appendix, if there is too much data to fit in the actual text nicely.

For now, the Aalto logo variants are shown in Figure~\ref{fig:aaltologo}.

\begin{figure}
\begin{center}
 \begin{subfigure}[b]{\textwidth}
  {\selectlanguage{english}\AaltoLogoSmall{1}{!}{aaltoBlue}}
  \caption{In English}
 \end{subfigure}
 \begin{subfigure}[b]{\textwidth}
  {\selectlanguage{finnish}\AaltoLogoSmall{1}{''}{aaltoRed}}
  \caption{Suomeksi}
 \end{subfigure}
 \begin{subfigure}[b]{\textwidth}
  {\selectlanguage{swedish}\AaltoLogoSmall{1}{?}{aaltoYellow}}
  \caption{P\r{a} svenska}
 \end{subfigure}
\caption{Aalto logo variants}
\label{fig:aaltologo}
\end{center}
\end{figure}
\fi

\chapter{Requirements for research data management and publishing system}
\label{chapter:reqs}

\section{Functional Requirements for Research Data Publishing}

\tabcolsep=0.11cm
\begin{tabularx}{\textwidth}{| >{\raggedright}p{3cm} | >{\raggedright}p{3cm} | X |}
    \hline
    \textbf{Requirement} & \textbf{Metric}& \textbf{Rationale} \\
    \hline
    \rowcolor{Gray}
    The publishing platform can host and serve big files    &The file maximum size is in the order of tens of gigabytes & The platform does not need to serve big data, but even ''normal'' data can be gigabytes in size\\
    \hline
    Users can sort the search results by different factors &The search results can be sorted by various sorting criteria, e.g. dataset publication date, dataset language, dataset creation date & If the repository contains many datasets, searching for relevant ones needs to be made easier\\
    \hline
    \rowcolor{Gray}
    The metadata templates offered by the system satisfy the needs of different fields of science    &The metadata templates follow the existing metadata standards or best practices & Metadata should be descriptive of the data and understandable by others than the creators of the data\\
    \hline
    Data can be uploaded programmatically  &The platform offers and API with associated documentation  & If you possess a lot of data, you cannot upload it all manually\\
    \hline
    \rowcolor{Gray}
    Data can in the system can be visualized        &Data visualization follows the best practices of the field  & Good visualizations give an easy overview to the data\\
    \hline
    Access control to datasets should be fine grained  & Access to the datasets can be controlled by e.g. user account, groups of users, IP address range or embargoes & Sometimes you need the data to be only visible to a subset or there might be an embargo on the data – for example, students of Aalto University should see a certain dataset but not everyone in the world\\
    \hline
    \rowcolor{Gray}
    The data repository must be secure   &The data repository satisfies industry standard requirements for security& The data repository may contain confidential information which should be safe\\
    \hline
    Users in the system can be given different roles &The different roles are e.g. administrator, curator, contributor and guest & Publishing research data involves many parties, such as librarians and IT staff, and their roles are not about creating data but manage the system and curate the data\\
    \hline
    \rowcolor{Gray}
    The system should be integrated to the already existing user management system   &The system’s user management module is extensible to integrate external user management system & Not only do new students come in every year, staff changes all the time – and keeping multiple systems up to date is a cause for problems\\
    \hline
    Unknown vocabulary should be made clear to the users  &When encountered with unknown vocabulary, the user must get help within the same screen  & Research data has a lot of vocabulary that is not all known for researchers – helping them to understand that is important\\
    \hline
    \rowcolor{Gray}
    Users can add tags to their datasets and files        &Tags can be added to both datasets and files & Tags allow the system to group similar datasets and files to help find relevant ones\\
    \hline
    Users can to filter search results & The filtering can be done by e.g. field of science, dataset size or creator of the data& If the repository contains many datasets, finding the relevant ones is easier from filtered results\\
    \hline
    \rowcolor{Gray}
    The system can be integrable with systems for long term archival of selected important datasets  &Selected research datasets can be stored for over 20 years in a long term archival system& Long term archival allows for the persistence of science and continued use of the datasets\\
    \hline
    The system enables different versions of the dataset &The system can store and display different versions of the dataset & Datasets can be worked on and changed, but the old versions should be available since someone might use the older versions of the data\\
    \hline
    \rowcolor{Gray}
    The system allows for downloadable citations to the dataset   &The system gives citations in common formats & Citing datasets is much easier when you can get citations right out of the system\\
    \hline
\end{tabularx}

\pagebreak

\section{Functional Requirements for Research Data Management}

\tabcolsep=0.11cm
\begin{tabularx}{\textwidth}{| >{\raggedright}p{3cm} | >{\raggedright}p{3cm} | X |}
    \hline
    \textbf{Requirement} & \textbf{Metric}& \textbf{Rationale} \\
    \hline
    \rowcolor{Gray}
    Research data management tool must not interfere with the research work    &Research data management must not take more than 5\% of researcher’s time & If the tool causes problems it is not going to be used\\
    \hline
    Publishing data from the research data management tool should be easy&It should take fewer than 5 interactions with the system to publish a dataset from the research data management tool &There is no point in separating the systems and making publishing easy it makes publishing more likely\\
    \hline
    \rowcolor{Gray}
    The tool must be able to store metadata in addition to storing the actual data  &The metadata in the system should follow the metadata standards of the publishing platform&Data without metadata is incomplete and making metadata a part of the research data management tool would promote metadata even during the research process\\
    \hline
    The tool should have a graphical user interface  &There is a graphical user interface to the system that is available through the Internet & Not all researchers like a command line interface\\
    \hline
    \rowcolor{Gray}
    The tool should have a command line interface        &There is a command line interface to the system that can be accessed for example with SSH & Not all researchers like graphical user interfaces\\
    \hline
    The tool should be integrable to data collection devices  & Once integrated, the users don’t need to manually do anything to upload the raw data to the system & The more computerized the research process is the better it is for the researchers\\
    \hline
    \rowcolor{Gray}
    The system can be integrated to research workflow systems   &The system offers APIs to integrate it to existing workflow systems& Some research projects have research workflows that automate the research process and that should be accommodated\\
    \hline
    The system allows for comments on shared files &The system has a commenting function for the objects in the system & Comments help with collaboration on the files\\
    \hline
    \rowcolor{Gray}
    The system uses the existing user management system &The system integrates to the existing user management system of the institution &We don’t want to add more user accounts or points of confusion to the user\\
    \hline
    The research data should be accessible from anywhere  &The system does not filter traffic based on an IP address range, but the system should use the user management system to authenticate users  & Researchers might want to work from home, for example\\
    \hline
\end{tabularx}

\section{Hardware Requirements}

\tabcolsep=0.11cm
\begin{tabularx}{\textwidth}{| >{\raggedright}p{3cm} | >{\raggedright}p{3cm} | X |}
    \hline
    \textbf{Requirement} & \textbf{Metric}& \textbf{Rationale} \\
    \hline
    \rowcolor{Gray}
    Data is stored within the legal geographical limits &The hardware of the system is located such that the legal constraints are satisfied (for example, in the case of Finland the hardware is in Finland) & Some data cannot leave borders of countries\\
    \hline
    The data must be backed up for disaster recovery&The data storage is designed following good standards and principles &The research data is sensitive and should be backed up in case of failures – full back up on tapes is not feasible, but good enough replication is required\\
    \hline
    \rowcolor{Gray}
    The hardware must adhere to the safety requirements of the data &The hardware conforms to the safety standards of the data&Some data must be safer than other data – however, the most extreme cases should be handled separately\\
    \hline
    The hardware should be virtualized by the software on top of it &The hardware can be changed and the system on top of it does not need to be changed & The research data lifespan is likely to outlive several compute and storage hardware generations and hardware will always eventually fail\\
    \hline
\end{tabularx}
