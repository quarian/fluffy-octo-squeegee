All fields of science are becoming data intensive. The decrease of computing
price and the evolution of data collection methods have opened new avenues of
analysis that have not been possible before. This new data intensive paradigm
puts a new kind of premium to research data, since it more than ever before
forms the lifeblood of research. At the same time the demand for publishing
research data has increased from both funding bodies and the research
community. These factors combined present novel challenges for managing
ans publishing research data.

This thesis sheds light on the current status of research data management,
sharing and publishing. The primary contribution of the thesis is the
examination of existing techincal solution to these research data challenges.
The secondary contribution of the thesis is the research of the cultural
atmosphere surrounding research data management and sharing. The cultural
research also yields a view of the current affairs in Finland and
internationally in this field.

Techincal solutions for sharing and managing research data were found from
within the open source community. Solutions like Dataverse, Invenio, Hydra
Project and CKAN offer platforms for sharing and publishing data. Solutions
like iRODS can be used to manage research data. These technical solutions serve
their purpose, but there is no good integrated solution for both managing and
sharing research data. There are, however, techincal and user experience
related improvements that could be done.

The biggest barrier to sharing and managing research data is, however, the
lack of culture and know-how on the topic. Future work should, in addition to
building an integrated solution for sharing and managing research data, focus
on how to change the culture of research data management to be more open. The
benefits of opennes of data outweigh the work that goes into making research
data public and that should be the goal for future research.


%A dissertation or thesis is a document submitted in support of candidature
%for a degree or professional qualification presenting the author's research and
%findings. In some countries/universities, the word thesis or a cognate is used
%as part of a bachelor's or master's course, while dissertation is normally
%applied to a doctorate, whilst, in others, the reverse is true.
%
%d\fixme{Abstract text goes here.}
%
%\texttt{\textbackslash fixme\{\}} is a command that helps you identify parts of
%your thesis that still require some work.
%When compiled in the custom \texttt{mydraft} mode, text parts tagged with
%fixmes are shown in bold and with fixme tags around them.
%When compiled in normal mode, the fixme-tagged text is shown normally (without
%special formatting).
%The other draft mode command is \texttt{\textbackslash TODO\{\}}
%which precedes its argument with a coloured
%\TODO{, encloses it in brackets and types it in bold face.}
%In normal mode it produces nothing.
%
%The draft mode also causes the ``DRAFT'' text to appear on
%the front page, alongside with the document compilation date. The custom
%\texttt{mydraft} mode is selected by the \texttt{mydraft} option given for the
%package \texttt{aalto-thesis}, near the top of the \texttt{thesis-example.tex}
%file.
%
%The thesis example file (\texttt{thesis-example.tex}), all the chapter content
%files (\texttt{1introduction.tex} and so on), and the Aalto style file
%(\texttt{aalto-thesis.sty}) are commented with explanations on how the Aalto
%thesis works. The files also contain some examples on how to customize various
%details of the thesis layout, and of course the example text works as an
%example in itself. Please read the comments and the example text; that should
%get you well on your way!
