Kivi on materiaali, joka muodostuu mineraaleista ja luokitellaan
mineraalisis{\"a}lt{\"o}ns{\"a} mukaan. Kivet luokitellaan yleens{\"a} ne muodostaneiden
prosessien mukaan magmakiviin, sedimenttikiviin ja metamorfisiin kiviin.
Magmakivet ovat muodostuneet kiteytyneest{\"a} magmasta, sedimenttikivet vanhempien
kivilajien rapautuessa ja muodostaessa iskostuneita yhdisteit{\"a}, metamorfiset
kivet taas kun magma- ja sedimenttikivet joutuvat syv{\"a}ll{\"a} maan kuoressa
l{\"a}mp{\"o}tilan ja kovan paineen alaiseksi.

Kivi on ep{\"a}orgaaninen eli elottoman luonnon aine, mik{\"a} tarkoittaa ettei se
sis{\"a}ll{\"a} hiilt{\"a} tai muita elollisen orgaanisen luonnon aineita. Niinp{\"a} kivest{\"a}
tehdyt esineet s{\"a}ilyv{\"a}t maaper{\"a}ss{\"a} tuhansien vuosien ajan m{\"a}t{\"a}nem{\"a}tt{\"a}. Kun
orgaaninen materiaali j{\"a}tt{\"a}{\"a} j{\"a}lkens{\"a} kiveen, tulos tunnetaan nimell{\"a} fossiili.

Suomen peruskallio on suurimmaksi osaksi graniittia, gneissi{\"a} ja
Kaakkois-Suomessa rapakive{\"a}.

Kive{\"a} k{\"a}ytet{\"a}{\"a}n teollisuudessa moniin eri tarkoituksiin, kuten keitti{\"o}tasoihin.
Kivi on materiaalina kalliimpaa mutta kest{\"a}v{\"a}mp{\"a}{\"a} kuin esimerkiksi puu.
