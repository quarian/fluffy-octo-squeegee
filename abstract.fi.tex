Tutkimustieto on nykyp{\"a}iv{\"a}n{\"a} keskeinen osa kaikkea tutkimusta. Tieteellisen
laskennan hinnan lasku ja tutkimustiedon ker{\"a}{\"a}mismenetelmien kehitys ovat johtaneet
uusiin tutkimusmenetelmiin. Tutkimustietokeskeinen suuntaus asettaa
tutkimustiedon t{\"a}rke{\"a}mp{\"a}{\"a}n asemaan kuin koskaan aiemmin, sill{\"a} ilman
laadukasta tutkimustietoa parhaat tulokset j{\"a}{\"a}v{\"a}t saavuttamatta.
T{\"a}m{\"a}n johdosta tutkijayhteis{\"o} ja tutkimuksen rahoittajatahot ovat
alkaneet vaatia tutkimustiedon julkaisemista. N{\"a}iden kehitysten johdosta
tutkimsudatan hallintaan ja julkaisuun t{\"a}ytyy kehitt{\"a}{\"a} uusia ratkaisuja.

T{\"a}m{\"a} opinn{\"a}ytety{\"o} valottaa tutkimusdatan julkaisemisen, jakamisen ja
hallinnan nykytilannetta. Opinn{\"a}ytteen p{\"a}{\"a}panos on tutkimusdatan
haasteita ratkovien teknisten toteutusten tutkimus, mink{\"a} lis{\"a}ksi ty{\"o}ss{\"a}
ehdotetaan onnistuneen tutkimusdataratkaisun m{\"a}{\"a}ritelm{\"a}{\"a}. Opinn{\"a}ytety{\"o}
my{\"o}s tutkii tutkimustiedon julkaisemisen ja hallinnan kulttuuria ja
k{\"a}yt{\"a}nt{\"o}j{\"a}.

Opinn{\"a}ytety{\"o}ss{\"a} esitell{\"a}{\"a}n avoimen l{\"a}hdekoodin ratkaisuja
tutkimusdatan jakamiseen ja hallintaan. Dataversen, Invenion, Hydra-projektin
ja CKANin kaltaiset j{\"a}rjestelm{\"a}t ovat alustoja tutkimusdatan julkaisemiseen.
iRODSin kaltaiset sovellukset soveltuvat tutkimusdatan hallintaan. N{\"a}m{\"a}
ratkaisut toimivat, mutta yhdistetty{\"a} ratkaisua tutkimustiedon hallintaan
ja jakamiseen ei ole. Kokonaisratkaisujen sek{\"a} tutkimustiedon
hallintaan ja jakamiseen liittyv{\"a}n kulttuurin puutteesta seuraa, ett{\"a}
tutkimusdataa jaetaan v{\"a}h{\"a}n verrattuna sen m{\"a}{\"a}r{\"a}{\"a}n.

Jatkotutkimuksen tulisi keskitty{\"a} tutkimusdatan hallinnan ja julkaisemisen
yhdist{\"a}v{\"a}n palvelun lis{\"a}ksi etsim{\"a}{\"a}n ratkaisuja aiheeseen liittyv{\"a}n kulttuurin ja
tietotaidon parantamiseen.

\iffalse
Kivi on materiaali, joka muodostuu mineraaleista ja luokitellaan
mineraalisis{\"a}lt{\"o}ns{\"a} mukaan. Kivet luokitellaan yleens{\"a} ne muodostaneiden
prosessien mukaan magmakiviin, sedimenttikiviin ja metamorfisiin kiviin.
Magmakivet ovat muodostuneet kiteytyneest{\"a} magmasta, sedimenttikivet vanhempien
kivilajien rapautuessa ja muodostaessa iskostuneita yhdisteit{\"a}, metamorfiset
kivet taas kun magma- ja sedimenttikivet joutuvat syv{\"a}ll{\"a} maan kuoressa
l{\"a}mp{\"o}tilan ja kovan paineen alaiseksi.

Kivi on ep{\"a}orgaaninen eli elottoman luonnon aine, mik{\"a} tarkoittaa ettei se
sis{\"a}ll{\"a} hiilt{\"a} tai muita elollisen orgaanisen luonnon aineita. Niinp{\"a} kivest{\"a}
tehdyt esineet s{\"a}ilyv{\"a}t maaper{\"a}ss{\"a} tuhansien vuosien ajan m{\"a}t{\"a}nem{\"a}tt{\"a}. Kun
orgaaninen materiaali j{\"a}tt{\"a}{\"a} j{\"a}lkens{\"a} kiveen, tulos tunnetaan nimell{\"a} fossiili.

Suomen peruskallio on suurimmaksi osaksi graniittia, gneissi{\"a} ja
Kaakkois-Suomessa rapakive{\"a}.

Kive{\"a} k{\"a}ytet{\"a}{\"a}n teollisuudessa moniin eri tarkoituksiin, kuten keitti{\"o}tasoihin.
Kivi on materiaalina kalliimpaa mutta kest{\"a}v{\"a}mp{\"a}{\"a} kuin esimerkiksi puu.
\fi
