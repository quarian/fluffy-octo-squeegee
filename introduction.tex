\chapter{Introduction}
\label{chapter:intro}

The world of science is moving towards more and more data intensive research.
Methods for gathering research data and analyzing the data are growing
cheaper and cheaper while the knowledge of algorithms and statistical analysis
keeps improving. This has made fields that were previously very data intensive,
such as particle physics, accumulate even more data. On the other hand
fields, such as social sciences, where the amounts of data have
traditionally been small to quickly become much larger. This new world sets
research data to a new kind of premium - it is the heart of research more
than ever before in the past. \cite{DBLP:books/ms/4paradigm09}

With the increased value of research data managing that data becomes important.
At the same time the world is becoming more and more connected, which enables research
data to be transferred easily all over the world. A big
part of research nowadays is done in groups that are scattered all around the
world, which means that sharing research data with your partners becomes a real challenge. 
Research data might be too big to be sent via email or other traditional tools
or it might contain data that can not leave a secure datacenter.
\cite{DBLP:journals/jbi/HarrisTTPGC09, DBLP:journals/jasis/Borgman12}

The advance of technology has also made it possible to share data with anyone,
and since data generated by researchers is often done with public funding there
is a logical argument to be made that publicly funded research data should be
available for the public good as well. \cite{DBLP:journals/jasis/Borgman12} This imposes challenges for the
researchers, since in order to make research data public and useful for others
appropriate metadata needs to be added to the data. This process is both
time consuming and not necessarily required to do research, making it a low
priority for researchers \cite{savage2009empirical}.

This thesis tackles the problems of the new, data intensive world of research
with a focus on the technical implementations to research data publication.
The research questions are defined in Section \ref{chapter:problem} along. Section
\ref{chapter:problem} also outlines
the approach of this thesis and its main contributions.

In Section \ref{chapter:background} scientific literature about research data
management and sharing, open access publishing and other relevant fields are
presented. The literature shows that open research data and openly accessible
research papers further science, but there are many challenges that go into
the process of opening them up. The problems of sharing research data are not
only technical, but organizational and cultural matters have a big impact. The
literature review also covers related areas, such as linked data, research
data policy, research data curation and research workflows.

Section \ref{chapter:positioning} positions the thesis in the Finnish field and
presents a snapshot of the current level of research data management and
sharing in Finland. The chapter contains interviews from the multiple
stakeholders that deal with research data, questionnaires that were used to
gauge the needs of Aalto University employees regarding research data and
benchmarking of existing technical solutions. The findings echo the findings from the
literature review in that there is little know-how or culture towards research
data publishing and research data management is handled with very different
approaches across individuals. Functional technical solutions exist and they
seem to fill their roles, but a holistic solution that would take care of both
managing research data during the research process and publishing it in the end
is both missing and needed. The results from the multiple stakeholder groups
interviewed also strengthen the view that simply techincal solutions do not
solve the research data management and publication problem - there is a need to
teach research research data management and publshing to change the perceptions
and culture surrounding them.

The positioning research leads to the experimental part of the thesis. Section
\ref{chapter:prototype} describes how we chose Harvard Dataverse from the
group of existing technical solutions and how we use it to gain further insights about
the technical implementations of research data publishing platforms. The user tests
show that the mechanical process of using the research data publication
platform is easy to learn and use, but there are some technical and usability
improvements that could be made. Interactions with the users also brought up
the point again that while the solution does seem to work, it would be hard
for them to use since their research data is not primed for publication. In
this context the solution working means that it can be used to upload, search
and download research data.

The research done for this thesis is discussed in Section
\ref{chapter:discussion}. In addition to evaluating the research methods
some synthesis and suggestions for future research and steps in the area
are suggested. Future work should include investigation into how a
holistic solution for managing and publishing research data could be
implemented as well as investigation on how the culture of research data
management could be opened up.

The conclusions of the thesis are presented in Section
\ref{chapter:conclusions}. The section shows all the studied technical
implementations in comparison with regards to research data publishing
and management. This study has also yields many requirements and points
that need to be taken into consideration when designing or evaluating a
research data management and publishing system. These requirements are
presented for future use. As for the technical implementations, it is
concluded that while they do sufficiently well in what they are designed to
do, there is a need for a solutions that would combine both research data
management and publishing. It is also concluded that in order to make these
technical solutions work and spread, the culture of research data management and sharing
has to be made more open.

\iffalse
This is the introduction chapter - it will contain the following things.

\begin{itemize}
    \item general introduction to the subject
    \item general background
    \item objectives for the maste'r thesis
    \item the main research question - how to share research data?
    \begin{itemize}
        \item how to do it?
        \item how is it being done at the moment=
    \end{itemize}
    \item subproblems
    \begin{itemize}
        \item technical
        \item cultural
        \item organizational
    \end{itemize}
    \item what this thesis covers and what it does not cover
    \item positioning the work and how it's connected to other work
    \item important concepts
    \item the structure of the thesis
\end{itemize}
\fi
