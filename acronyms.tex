\chapter*{Terminology}
\phantomsection
\addcontentsline{toc}{chapter}{Terminology}


% The longtable environment should break the table properly to multiple pages, 
% if needed

\noindent
\begin{longtable}{@{}p{0.25\textwidth}p{0.7\textwidth}@{}}
Research data & In the context of this thesis, research data refers to data that
has been generated or used in scientific work\\
Research papers & Research papers is the umbrella term used in this thesis to cover
traditional scientific publishing material, such as journal articles and conference papers\\
Research data management & Research data management refers to to the act of managing research
data during a research project - this includes but is not limited to documenting, annotating and
cleaning research data \\
Research data sharing & Research data sharing refers to sharing research data between two parties,
either by sharing it privately or making the research data available \\
Research data publishing & Research data publishing refers to the act of making research data
public for all the world to see \\
Metadata & Metadata is descriptive data about data that is used to give context and other
additional information about the data itself\\ 
CSC  & CSC is a Finnish provider of scientific computing and storage services\\ 
EUDAT& EUDAT is a EU level initiative that aims to bring research data management, sharing and
publishing tools to European research institutions\\ 
DOI & DOI (Digital object identifier) is a commonly used persistent identifier scheme for research papers \\
Handle & Handle is a persistent identifier scheme \\
URN & URN is a Finnish persistent identifier scheme \\
Dataverse & Dataverse is an open source research data publishing platform originally developed in Harvard University \\
iRODS & iRODS is an open source research data management software \\
Etsin & Etsin is a metadata publishing tool for Finnish institutions \\
Avaa & Avaa is a Finnish service to open datasets for public use \\
ATT & ATT (Avoin Tiede ja Tutkimus) is a Finnish ministry led initiative to introduce openness to the Finnish field of science \\
PAS & PAS (Pitkäaikaissäilytys) is a Finnish project to develop long term archival of datasets \\
Hydra Project & Hydra Project (also referred to as Hydra in this thesis) is an extensible open source repository solution \\
Invenio & Invenio is a CERN based, now open source data repository solution \\
Zenodo & Zenodo is a public instance of Invenio with a custom user interface \\
GitHub & GitHub is a platform for collaborating on and sharing source code \\
B2Share & B2Share is a research data publishing service of the EUDAT initiative \\
B2Drop & B2Drop is a research data sharing and managing tool of the EUDAT initiative \\
Apache Solr & Apache Solr (also referred to as Solr in this thesis) is a tool that in the context of data repositories is often
used to index databases to make them searchable \\
ACRIS &  ACRIS (Aalto Current Research Information System) is a system to manage research information being implemented in Aalto University.
It incorporates Elsevier Pure, a tool that has research data management and publishing features

\end{longtable}
