\chapter{Positioning the Thesis}
\label{chapter:positioning}

In this chapter the thesis is positioned to the existing landscape.

\section{Expert interviews}
\label{sec:expert_interviews}

We have data from multiple expert interviews, so we'll present them here.

\begin{itemize}
    \item many experts
        \begin{itemize}
            \item scientists
            \item science IT staff
            \item people in charge of policies and development of university
                  systems
            \item developers in libraries
            \item library representatives
            \item project managers at CSC (in charge of nationwide computing
                  services)
        \end{itemize}
        \item fill in input from each once it's time to write those
        \item notable common findings
        \begin{itemize}
            \item metadata management is a huge problem
         demand to publish research data   \item no education, tools or any idea on how to work woth that
            \item no experience in data sharing
            \item not willing to use lots of time for metadata management
            \item the university would prefer not to store their own
                  datasets, but would rather work with other places
            \item library should have a role in managing metadata anc coaching
                  the scientist with their dataset management
            \item data intensive science is being taught in Aalto, but without
                  own infrastructure
        \end{itemize}
        \item also a common conclusion here is that there is not concesus
              within Finland on how to arrange ourselves with research data
        \item there is the nationwide computing service provider, CSC, but
              there is no common push towards data repository or managing
              research data
\end{itemize}

\section{Data management questionnaire}
\label{sec:questionnaire}

Anne Sunikka provided a quesitonnaire about what is required from data
management, so we can use it here. Look through it later and summarize the
information available here.

\section{Benchamrking existing solutions}
\label{sec:benchamrking}

And of course solutions exist already, so we'll briefly go over them here.


\section{Work that is currently going on in Finland}
\label{sec:finland_current}

Here we are going to paraphrase research and interesting things going on in
Finland at the moment. Items to look into would be:

\begin{itemize}
    \item IDA and Etsin services by CSC
    \item the ATT initiative going on
    \item the social science library by Tampere University
    \item the long term storage developed by national library
    \item national library also develops the URN schemes
\end{itemize}

A question to ask Keijo later is that many of the people interviewed told about
their upcoming research. Is it interesting here or should we stick to things
that exist for sure?

\section{Outcomes of the positioning research}
\label{sec:positioning_outcomes}

Here we will describe what we learned about the positioning of the work.

Research data management questions transcend the university level as well
as the national borders. After all, the goal of publishing and sharing
researach data is to make science move forward in a faster clip and make the
quality of research better by collaboration - ergo there is no point in every
institution in Finland to manage their own research data repositories and
archives.

With this in mind this thesis focuses on providing a prototype solution that
serves as a wireframe for implementation that should be done in such a way that
all institutions in Finland would benefit from that. In addition this research
will provide a roadmap style solution that would make use of existing know-how
and work and serve as a possible proposal on how to bring research data
management into all Finnish institutions that need that kind of service.
