\chapter{Positioning the Thesis}
\label{chapter:positioning}

In this chapter the thesis is positioned to the existing landscape. TODO:
figure out a proper way to reference interviews.

\section{Expert interviews}
\label{sec:expert_interviews}

For the thesis many experts from different positions were interviewed. The goal
of the interviews was to get a more current look into the world of research
data sharing as well as establish what could be the role of Aalto Univrersity
as well as universities in general.

\section{Scientists}

Scientists would be the main user or stakeholder that would use the system. It
is important that a system that would help them store and publish research data
would provide value for them without being another system that is a huge
annoyance to use.

\subsection{The goal of the interviews}

It is interesting as to what is the current status of research data management
and how research data is being published. Also what was learned in interviews 
with other stakeholders it became clar that research data management covers
much more than just the publishing part, so it was important to find out
whether the scientists were actually being trained to handle research data.

The secondary goal is also to gain input on how the system to manage and
publish research data should look like if it were to be implemented.

\subsection{Results of the interviews}

\section{Science IT staff}

If there was to be a system to manage and publish research data, it would have
to be maintained by someone. The obvious answer is to go to the administrators
of institutional software infrastructure.

\subsection{The goal of the interviews}

What is the stance of the administrative side on a data repository? How does it
fit the role of the infrastructure administrators? Building software systems is
not only about what people want and what existing systems prove to be the best,
human factors and elements like funding affect these decisions.

At least in Aalto the staff maintaining the infrastructure are also aware about
who are the people hosting research data on school infrastructure, making them
also knowledgeable on how things are handled right now.

\subsection{Results of the interviews}

\section{Project managers on research data related systems}

Implementation and running of different systems related to research data
management and publishing are being implemented and developed around Finnish
institutions. While people working on and with these systems were included in
the previous two sections (backwards reference) there are people who are in
charge of those systems and make the decisions on which to purchase.

\subsection{The goal of the interviews}

Systems that have to do with entities as big as universisites or even nations
need management and getting a view from only the people using them is not
sufficient when designing such a system. People in managerial positions have
different objectives and needs for a system like this.

From people in positions like this it is importan to learn about the bigger
picture. When dealing with a concept like sharing research data it makes sense
to try make systems talk to each other and not everyone to implement their own
system in their own silo.

\subsection{Results of the interviews}

\section{Librarians}

Librarians are metadata experts. They are trained to describe scientific
content, manage and sort that content and nowadays they also work with digital
publishing. In some libraries across the globe (citation) the libraries have
taken responsibility of also publishing research data.

\subsection{The goal of the interviews}

What is the current status of digital publishing in Finland? And how do the
libraries themselves see their role when publishing research data enters the
equation. How should we organize the collaboration between libraries and
scientists?

\subsection{Results of the interviews}

\section{People organizing courses}

Teaching is another role within the university that would benefit from the fact
that research data would be readily available for the students to consume and
run analysis on. If we wanted to make research data management an integral part
of scientists' daily routine it would have to be integrated to the teaching
regime taught in schools.

\subsection{The goal of the interviews}

Aalto University has a minor in Data Science, but does it have any means to
support that? How is data science and in its wake data management taught
currently in the university?

\subsection{Results of the interviews}

\section{Representatives from CSC}

CSC offers an interesting angle to the research data management problem. CSC
offers nationwide computing services and they should have a role in sharing and
managin researhc data.

\subsection{The goal of the interviews}

CSC must have some plans and current ideas on how to manage and share research
data since the world is moving towards publishing research data. It also makes
sense from their point of view to offer both computing power and storage
services and it would be interesting to know what are their plans in that
sector and if they don't have plans what is going on with that.

\subsection{Results of the interviews}

\section{}

\begin{itemize}
    \item many experts
        \begin{itemize}
            \item scientists
            \item science IT staff
            \item people in charge of policies and development of university
                  systems
            \item developers in libraries
            \item library representatives
            \item project managers at CSC (in charge of nationwide computing
                  services)
        \end{itemize}
        \item fill in input from each once it's time to write those
        \item notable common findings
        \begin{itemize}
            \item metadata management is a huge problem
         demand to publish research data   \item no education, tools or any idea on how to work woth that
            \item no experience in data sharing
            \item not willing to use lots of time for metadata management
            \item the university would prefer not to store their own
                  datasets, but would rather work with other places
            \item library should have a role in managing metadata anc coaching
                  the scientist with their dataset management
            \item data intensive science is being taught in Aalto, but without
                  own infrastructure
        \end{itemize}
        \item also a common conclusion here is that there is not concesus
              within Finland on how to arrange ourselves with research data
        \item there is the nationwide computing service provider, CSC, but
              there is no common push towards data repository or managing
              research data
\end{itemize}

\section{Data management questionnaire}
\label{sec:questionnaire}

Anne Sunikka provided a quesitonnaire about what is required from data
management, so we can use it here. Look through it later and summarize the
information available here.

\section{Benchamrking existing solutions}
\label{sec:benchamrking}

And of course solutions exist already, so we'll briefly go over them here.


\section{Work that is currently going on in Finland}
\label{sec:finland_current}

Here we are going to paraphrase research and interesting things going on in
Finland at the moment. Items to look into would be:

\begin{itemize}
    \item IDA and Etsin services by CSC
    \item the ATT initiative going on
    \item the social science library by Tampere University
    \item the long term storage developed by national library
    \item national library also develops the URN schemes
\end{itemize}

A question to ask Keijo later is that many of the people interviewed told about
their upcoming research. Is it interesting here or should we stick to things
that exist for sure?

\section{Outcomes of the positioning research}
\label{sec:positioning_outcomes}

Here we will describe what we learned about the positioning of the work.

Research data management questions transcend the university level as well
as the national borders. After all, the goal of publishing and sharing
researach data is to make science move forward in a faster clip and make the
quality of research better by collaboration - ergo there is no point in every
institution in Finland to manage their own research data repositories and
archives.

With this in mind this thesis focuses on providing a prototype solution that
serves as a wireframe for implementation that should be done in such a way that
all institutions in Finland would benefit from that. In addition this research
will provide a roadmap style solution that would make use of existing know-how
and work and serve as a possible proposal on how to bring research data
management into all Finnish institutions that need that kind of service.
