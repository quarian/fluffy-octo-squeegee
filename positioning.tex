\chapter{Positioning the Thesis}
\label{chapter:positioning}

Master's theses don't live in a vacuum. To position the thesis and provide the
best possible outcome for Aalto University we made an effort to find out what
is the current state of research data publishing and sharing as well as what
are the current challenges and projects in the Finnish landscape. While the
literature review in Section \ref{chapter:background} covered the overall view
of the current status of publishing and sharing research data and research
papers, the goal of this thesis is to provide value in the context of Finland.
The tools used to position the thesis were interviews, benchmarking existing
solutions and a questionnaire.

\section{Researcher interviews}
\label{sec:expert_interviews}

Scientists and researchers are the core user group of any publishing or sharing
system since they are the ones generating the data to the system and populating
the system. The research also shows that a successful repository system requires
user engagement \cite{DBLP:conf/ercimdl/Martinez-UribeM09}. Scientist and
researcher interviews were conducted within different research groups and
researcher in Aalto Otaniemi campus. The goal of these interviews was to learn
how data management is taught and used in Aalto University and what are the
scientist and researchers perceptions on sharing and publishing research data.
Previous experiences with sharing and using others' data were also gathered.

In an interview with a research group\footnote{M. Nurmela and the Complex
Networks group at Aalto University, \url{http://becs.aalto.fi/en/research/complex\_networks},
personal communication, July 31st, 2015} the findings fell in line with the
findings from literature. As of writing of this thesis, data management is not
systematically taught for the researchers and
there is no culture or experience in data sharing. Upon questioning it became
clear that the data in the possession of the researchers would have required
a lot of cleaning and metadata additions prior to publishing - this is no surprise
considering the fact that the datasets they were using were not designed to
go public in the first place. Though lack of metadata practices and lack of
publishing infrastructure were also brought up.

Some members of the research group had shared some of their datasets through
public cloud services such as Google Drive\footnote{\url{http://drive.google.com/}}
with collaborators and used others' datasets. This also raised the point that
in order to use others' datasets they had been asked to cite the papers that
used that dataset, underlining the the undeveloped culture of research data
sharing. Some members of the group saw big advantages in making research output
public, especially from the angle of reproducability, but also raised concerns
about the privacy issues that for example telecommunication data would cause.
The members of the group were also concerned about about the size of their
data, since using the existing solutions to share hundreds of gigabytes worth
of datawould prove cumbersome.

One member of the group had taken a role of a mentor with the data management
issues, teaching the others to use databases and version control software to
handle their data and code. The other members of the group, interviewed
separately, noted that an effort had been made towards better policies in data
handling. The group pointed out that personalized assistance and word of mouth
were an efficient way to learn "boring" things like data management. Own
previous experience and learning by doing seemed to be the main way people had
learned to, for example, comment their code or arrange their research data into
databases so that accessing them would require less time and managing code
would be easier.

Discussions with the complex networks groups also brought up the point that
even though some data could be published for all the world to see, some data
should be only accessible to people within Aalto University and some should
even have a more fine grained access control set to them.

In a separate interview with a brain image researcher\footnote{M. Nurmela and
E. Glerean, personal communication, August 13th, 2015} similar concerns arose.
Brain imaging data is large and that makes sharing it hard. Brain images are
also considered personal data and publishing them is problematic. The
researcher expressed interest in sharing and using others' datasets, but
lacked the tools to do so.

\iffalse
\section{Scientists}

Scientists would be the main user or stakeholder that would use the system. It
is important that a system that would help them store and publish research data
would provide value for them without being another system that is a huge
annoyance to use.

\subsection{The goal of the interviews}

It is interesting as to what is the current status of research data management
and how research data is being published. Also what was learned in interviews 
with other stakeholders it became clear that research data management covers
much more than just the publishing part, so it was important to find out
whether the scientists were actually being trained to handle research data.

The secondary goal is also to gain input on how the system to manage and
publish research data should look like if it were to be implemented.

\subsection{Results of the interviews}

\fi

\section{Science IT staff}
\label{sec:sci_it}

If a system to publish and share research data would come to be it would have
to be maintained and run by people other than the researchers, since the job of
the researchers is to do research and not maintain software systems. With this
in mind the people managing the scientific IT systems are a key component to
building a successful research data management, publication and sharing
platform.

In interviews with a scientific IT systems specialist\footnote{M. Nurmela and
M. Hakala, personal communication, July 1st and 7th, 2015} the findings again
aligned with the findings from literature. There is a lack of metadata
standards that would make data storage and management uniform across
institutions and disciplines. Finland has projects going on related to open
science\footnote{\url{http://avointiede.fi/}} and CSC\footnote{\url{https://www.csc.fi/}} is
building scientific computing environments for Finnish institutions (research,
library, archival, education and such). According to the specialist
collaborating with all these ongoing projects would benefit both parties and
also allow Alto University to develop systems that are not just point
solutions. This would also make sense from a financial point and practical
point of view, since research is nowadays done in collaboration with other
institutions and working together would enable that.

From the point of view of IT system specialist the ideal system for research
data includes the whole lifecycle of the data. This entails education on how
to manage the data from the creation to the publishing and infrastructure that
can be tailored to fit the different user needs. Research data is comes in
many forms so a solution that is not flexible would be unsuitable.

University of Jyväskylä has implemented a data repository system\footnote{\url{https://dvn.jyu.fi/dvn/}}
as well as an iRODS\footnote{\url{http://irods.org/}} system to manage and
store research data. In an interview\footnote{M. Nurmela, I. Korhonen and A.
Auer, personal communication, August 19th, 2015} they noted that nowadays
universities need a platform to publish research data. Jyväskylä is among the
first in Finland to implement a data policy in practice, offering an
infrastructure to manage data during the research projects and publish the
results in the end (even though Dataverse and iRODS are not currently
easily compatible with each other, though some work has been done for
that\footnote{\url{https://irods.org/wp-content/uploads/2014/06/Odum-DFC-iRODS-Boston.pdf}}).
They are also working with CSC to get their iRODS system talking with the iRODS
system at CSC.

The Jyväskylä experts told that while the Dataverse system was easy to install
and modify to accept Jyväskylä University credentials, it still had a long way to go
before it was universally accepted within the university. At the time of
writing this, the Jyväskylä Dataverse has been running for approximately a
year and it contains a very small amout of datasets. Some research groups,
however, are using it manage their internal datasets. As to how to get the
more datasets into the system, they planned to continue educating about it
and making it that way a part of researchers' routine.

Jyväskylä was more involved in development of the iRODS system, having even
developed a system to facilitate collaboration between researchers
\cite{irodsinproceedings}. The Kanki system is a desktop interface to the iRODS
system that allows users to easily access and modify data stored in the iRODS
data grid. The Kanki system is not about publishing research data but data
management during a research project. iRODS also has federation capabilities,
meaning that two iRODS instances could be integrated such that you could
access the data from the other system. When writing this Jyväskylä and CSC
were planning to start testing the federation capabilities. Following up on that
on a later date would be interesting.

\iffalse
If there was to be a system to manage and publish research data, it would have
to be maintained by someone. The obvious answer is to go to the administrators
of institutional software infrastructure.

\subsection{The goal of the interviews}

What is the stance of the administrative side on a data repository? How does it
fit the role of the infrastructure administrators? Building software systems is
not only about what people want and what existing systems prove to be the best,
human factors and elements like funding affect these decisions.

At least in Aalto the staff maintaining the infrastructure are also aware about
who are the people hosting research data on school infrastructure, making them
also knowledgeable on how things are handled right now.

\subsection{Results of the interviews}

\fi

\section{Project managers on research data related systems}

Building and running software systems requires commitment not only from the
primary users discussed in Sections \ref{sec:expert_interviews} and
\ref{sec:sci_it} but the management that supports them. The priorities of the
managerial types might not lie in the usability or maintainability of the
system, but rather in managing costs and minimizing risks.

From Aalto side the desire is to minimize the systems we have to build and
maintain ourselves - since CSC exists to provide scientific computing and
storage resources, why should we not use them? Non established or new fields
of science could have value from a local repository, but those that have
international or discipline specific repositories could use those as well.\footnote{M. Nurmela, A. Sunikka, personal communication, July 17th, 2015}
Certainly managing research data is a problem and a consensus solution has not
emerged.

The Finnish National library is in charge of long term preservation of relevant
objects in the Finnish research. They are building a long term storage
solution\footnote{\url{http://avointiede.fi/tutkimus-pas}}, data management
plan tool\footnote{\sloppy\url{http://portti.avointiede.fi/tutkimusdata/tuuli-tyokalu-tutkimuksen-datanhallinnan-suunnitteluun}} and managing the Finnish unique identifier
service\footnote{\url{http://www.kansalliskirjasto.fi/fi/julkaisuala/urn.html}}.
They also manage a Finnish cultural repository\footnote{\url{https://www.finna.fi/}}.

The most important thing to for the project manager at the National Library
was that metadata associated with the data has to be good - otherwise
archival, management and reuse is impossible. Making metadata work within an
institution requires commitment from all levels of management and tools to
facilitate that. It is also important to note that even if two systems within
different organizations would be technologically perfectly compatible,
the bottlenecks might stem from different policies in different organizations
and the bureaucracy that comes with it. This thought lessens the burden for all
technology to be perfectly compatible.\footnote{M. Nurmela and E. Keskitalo, personal communication, August 21st, 2015}

\iffalse
Implementation and running of different systems related to research data
management and publishing are being implemented and developed around Finnish
institutions. While people working on and with these systems were included in
the previous two sections (backwards reference) there are people who are in
charge of those systems and make the decisions on which to purchase.

\subsection{The goal of the interviews}

Systems that have to do with entities as big as universities or even nations
need management and getting a view from only the people using them is not
sufficient when designing such a system. People in managerial positions have
different objectives and needs for a system like this.

From people in positions like this it is important to learn about the bigger
picture. When dealing with a concept like sharing research data it makes sense
to try make systems talk to each other and not everyone to implement their own
system in their own silo.

\subsection{Results of the interviews}

\fi

\section{Librarians}

Publishing research data requires expertise in digital publishing and metadata
experience. University libraries are experienced in publishing digital reserach
papers (open access or restricted access) and making metadata descriptions
about digital and physical publications. This makes librarians and libraries
and essential part in bringing research data to the open publishing world.

In an interview with the people responsible of the digital publishing at Aalto
University Library\footnote{M. Nurmela and A. Rousi, personal
communication, September 30th, 2015} brought up the essential role of
librarians as the classifiers and describers of the data. Professional data
handlers can do very good metadata descriptions, even if they are missing
some of the domain knowledge related to the research data. The librarians
also handle the relationships to the publishers - though what is the role
of traditional science publisher authorities in the future when organizations
can publish datasets and even research papers papers easily on their own.

Aalto University library runs the Aaltodoc service\footnote{\url{https://aaltodoc.aalto.fi/}}
which contains full text materials on research papers and theses published
in Aalto University. The system runs on on DSpace\footnote{\url{http://www.dspace.org/}},
an institutional repository software for publishing digital objects. DSpace
focuses on publishing research papers, but the person in charge of the system
reckoned that it probably could be modified to host small datasets as well.
This would require some additional work of course, so a better way could be to
link the research papers the relevant datasets in the corresponding systems.\footnote{M.
Nurmela and J. Nevala, personal communication, August 27th, 2015}

The National Library of Finland is in charge of implementing long term storage
and archival of important datasets and other research material. From that point
of view and also research data in general the biggest challenges are not
technical - software systems to store data and manage it exist, but making it
so that institutions themselves commit to managing and storing data is the
bigger challenge. And once different institutions are able to manage their own
data, the collaboration between the institutions' systems is likely to be
more difficult in the policy and bureucracy sense. There are also many
unresolved questions related to long term storage. Who decides what datasets
are used for the long term storage? What kind of metadata long term storage
requires in addition to the metadata already in the original dataset? What
is the most suitable file format for long term storage, since tools and
software used to create it outdated relatively soon? The work
to figure out these things and the Finnish long term archival project is going
on during the writing of this thesis.\footnote{M. Nurmela and E. Keskitalo, personal communication, August 21st, 2015}

\iffalse
Librarians are metadata experts. They are trained to describe scientific
content, manage and sort that content and nowadays they also work with digital
publishing. In some libraries across the globe (citation) the libraries have
taken responsibility of also publishing research data.

\subsection{The goal of the interviews}

What is the current status of digital publishing in Finland? And how do the
libraries themselves see their role when publishing research data enters the
equation. How should we organize the collaboration between libraries and
scientists?

\subsection{Results of the interviews}

\fi

\section{People organizing courses}

In addition to research, the mission of universities is to teach. With the
world of science moving to the direction of more and more data intensive
universities need to adapt and offer students the possibility to study with
relevant datasets. Aalto University has started offering a minor in Data
Science in order to cater to this need.\footnote{\url{http://studyguides.aalto.fi/minors-guide/2015/en/sci/sci-minors-for-all-aalto-students/analytics-and-data-science.html}}.

In an interview with the people in charge of the Aalto Data Science minor
it became clear that even thoug data invensive science is taught, there is no
Aalto infrastructure for the teaching. Datasets and the computing power are
acquired from vendors which had caused some awkward way arrangments since the
access to the outside resources had to be controlled more tightly than just
on Aalto level. These issues can be worked out though and from the point of
teaching it would be nice to have data available from within the university,
data and computing resources could also be acquired elsewhere.\footnote{M. Nurmela,
J. Bragge and P. Malo, personal communication, August 6th, 2015}

The people in charge of teaching also could offer insight to the question about
the basic skills that go to data analysis. The question is interesting since
in order to leverage the fact that science is becoming more and more data
intensive scientists need to possess skills both to analyze their data and
manage it in a sufficent way. The skillset that's required for data handling
and management is largely progrmming, since naturally the analysis is
computerized and things such as cleaning and preparing data for analysis is
most efficient when done programmatically. On the other hand, the data analyis
part asks for skills in algorithmics and statistics. When we take into account
that the Aalto Data Science minor, for example, is open for students from all
fields of science, it seems that we need to be teaching a new set of skills to
almost all students that want to take part into the new wave of data intensive
science.

\iffalse

Teaching is another role within the university that would benefit from the fact
that research data would be readily available for the students to consume and
run analysis on. If we wanted to make research data management an integral part
of scientists' daily routine it would have to be integrated to the teaching
regime taught in schools.

\subsection{The goal of the interviews}

Aalto University has a minor in Data Science, but does it have any means to
support that? How is data science and in its wake data management taught
currently in the university?

\subsection{Results of the interviews}

\fi

\section{CSC and national level things}

CSC is the national computing service provider for Finnish institutions. They
offer both computing power and disk space for Finnish institutions and they
are a state-owned non-profit organization.\footnote{\url{https://www.csc.fi/csc}}
Their role is interesting when it comes to research data managemnet and
publishing, since they already offer computing services to the relevant
institutions in Findland. After all, one goal of publishing and sharing
research data is to enable collaboration and involving the one institution
in Finland that offers services to all the other institutions makes sense in
that regard.

One of CSC's services related to reserach data management is the IDA
service\footnote{\url{http://avointiede.fi/ida}} that is specifically
designed to store research data and the related metadata. CSC also maintians
Etsin\footnote{\url{https://etsin.avointiede.fi/}}, a service to host
metadata related to research data as well as links to the location of thedata.
Etsin itself does not contain datasets.
These efforts fall under the national project of Open
Science\footnote{\url{http://openscience.fi/}} (Avoin Tiede ja Tutkimus in
Finnish) that promotes the openness of resesarch and science. On the top level
these initiatives have been put into motion by the Ministry of Education and
Culture\footnote{\url{http://www.minedu.fi/OPM/?lang=en}}.

The IDA service, however, has not been very widely adopted as a place to store
and share research data. This has to do with IDA's user interface and the fact
that the iRODS backend is not designed for publishing information. Issues with
policies and permissions also hinder the publication process.\footnote{M.
Nurmela and S. Westman, personal communication, August 14th, 2015} In addition
to the technical matters there is of course the fact that institutions, such as
unversities, do not have a very well developed culture for research data
management or publishing which also contributes to this.


\iffalse
CSC offers an interesting angle to the research data management problem. CSC
offers nationwide computing services and they should have a role in sharing and
managing research data.

\subsection{The goal of the interviews}

CSC must have some plans and current ideas on how to manage and share research
data since the world is moving towards publishing research data. It also makes
sense from their point of view to offer both computing power and storage
services and it would be interesting to know what are their plans in that
sector and if they do not have plans what is going on with that.

\subsection{Results of the interviews}

\begin{itemize}
    \item many experts
        \begin{itemize}
            \item scientists
            \item science IT staff
            \item people in charge of policies and development of university
                  systems
            \item developers in libraries
            \item library representatives
            \item project managers at CSC (in charge of nationwide computing
                  services)
        \end{itemize}
        \item fill in input from each once it's time to write those
        \item notable common findings
        \begin{itemize}
            \item metadata management is a huge problem
         demand to publish research data   \item no education, tools or any idea on how to work with that
            \item no experience in data sharing
            \item not willing to use lots of time for metadata management
            \item the university would prefer not to store their own
                  datasets, but would rather work with other places
            \item library should have a role in managing metadata and coaching
                  the scientist with their dataset management
            \item data intensive science is being taught in Aalto, but without
                  own infrastructure
        \end{itemize}
        \item also a common conclusion here is that there is not consensus
              within Finland on how to arrange ourselves with research data
        \item there is the nationwide computing service provider, CSC, but
              there is no common push towards data repository or managing
              research data
\end{itemize}

\fi

\section{Data management questionnaire}
\label{sec:questionnaire}

A study was conducted in Aalto University about data management. The goal of
the study was to find out the current needs as well as the current status of
research data management within Aalto University (TODO: figure out a good way
to cite this study). 87 people submitted answers to the study from all the
schools in Aalto. The breakdown of answers is shown in figure (reference to the
figure).

The study outlined the challenges within Aalto University. The answers asked
most for services in storing data, metadata, finding data, archiving, sharing
and backing up data. The challenges where the requirements span from are both
technological and policy related. In the list below the the answers are
compiled by percentages:

\begin{itemize}
    \item The name of the folder/file or the location of the folder/file is
          forgotten or poorly described, approx. 35\%.
    \item Ownership of the data, approx. 30\%.
    \item Not enough disk space, approx. 30\%.
    \item Version control, approx. 30\%.
    \item Non functional or corrupted equipment, 29\%.
    \item Sharing data with partners and collaborators, 24\%.
    \item Unwanted deletion of data, 22\%.
    \item Complicated user interface, approx. 20\%.
    \item Forgetting password, 17\%.
    \item Access rights, approx. 12\%.
    \item Failed backup, 8\%.
\end{itemize}

The study shows that there are indeed a need already in place for a system to
manage research data in Aalto as well as training and policies to support data
management through the lifecycle of data.

\section{Benchmarking existing solutions}
\label{sec:benchmarking}

Technical solutions to publishing and managing research data exist already, many
of them open source and free. Some of the solutions were mentioned in the
literature review in section \ref{sec:implementations_literature}. This section describes
those systems and some other in more practical detail.

\subsection{Harvard Dataverse}

\subsection{Hydra project}

\subsection{iRODS}

\subsection{CKAN}

\subsection{Invenio and Zenodo}

\subsection{GitHub}

\section{Work that is currently going on in Finland}
\label{sec:finland_current}

Here we are going to paraphrase research and interesting things going on in
Finland at the moment. Items to look into would be:

\begin{itemize}
    \item IDA and Etsin services by CSC
    \item the ATT initiative going on
    \item the social science library by Tampere University
    \item the long term storage developed by national library
    \item national library also develops the URN schemes
\end{itemize}

A question to ask Keijo later is that many of the people interviewed told about
their upcoming research. Is it interesting here or should we stick to things
that exist for sure?

\section{Outcomes of the positioning research}
\label{sec:positioning_outcomes}

Here we will describe what we learned about the positioning of the work.

First of all, it is clear that the research data management is not jut a
technical problem. Perfectly fine technical solutions to the problems of
publishing research data and managing your data in a safe way exist but the
real problem is to get people and organizations to use them.

Research data management questions transcend the university level as well
as the national borders. After all, the goal of publishing and sharing
research data is to make science move forward in a faster clip and make the
quality of research better by collaboration - ergo there is no point in every
institution in Finland to manage their own research data repositories and
archives.

With this in mind this thesis focuses on providing a prototype solution that
serves as a wireframe for implementation that should be done in such a way that
all institutions in Finland would benefit from that. In addition this research
will provide a road map style solution that would make use of existing know-how
and work and serve as a possible proposal on how to bring research data
management into all Finnish institutions that need that kind of service.
