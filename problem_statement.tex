\chapter{Problem Statement}
\label{chapter:problem}

\section{Research questions}

The primary research question of this thesis is how research data can be shared
using modern tools? The rationale behind the question is that due to the
value of research data, advance of sharing technologies and the requirements
to make research data public publishing research data is a current problem.

During the research it became clear that there are other than technical
problems that prevent sharing data within the scientific community. In light
of this this thesis also answers the question of what is preventing researchers
from publishing their research data.

This thesis focuses on the first question, but it is
clear that without answering and providing solutions to the second question
sharing research data is not going to happen on a local level and it is not
going to be univerally adopted. This thesis does not provide answers on how
this cultural problem might be solved, but some directions for future
research are proposed.

\section{The contributions of this thesis}

The main contribution of this thesis is the techincal examination of existing
software solutions to research data sharing. The most widespread tools are
benchmarked and from among them one solution, Harvard Dataverse, is examined
in further detail for deeper understanding of the functioning of these systems.
The existing research data publication platforms are fairly similar, which
makes examination like this appliccable to the other solutions as well.

In addition to the technical examination this thesis proposes a set of
requirements that could be used as a basis of designing a research data
management system or validate an existing system for research data management.
The requirements are derived from the background research and the user tests
conducted.

This thesis also provides an overview of the cultural problems that overshadow
research data management and publishing.

\section{Approach to the problem}

To address the research problems the following methods are used:

\begin{itemize}
    \item Literature review
    \item Expert interviews
    \item Questionnaires
    \item User tests
\end{itemize}

Literature review is used to gain background knowledge of research data
management and publishing. Expert interviews are used to position the thesis
in the Finnish field - there is no point in working on something similar that
is already being solved. The expert interviews also aim to see if the findings
from the literature apply in practice.

Aalto University has is currently forming a policy about research data
management and as a part of that project two questionnaires about the
current needs of research data management have been conducted. The results of
those quesitonnaires are presented as additional evidence for the needs and
current status of research data management.

Since the questionnaires already exist to shed light on the current needs user
tests were chosen as the method to extract most value towards designing a
system for research data management and publishing. The reasoning is that since
a system that relies this heavily on the users, in this case mostly the
research scientists, the users should be heard first and foremost. Contextual
interviews and tests with lead users were conducted to gain deeper
understanding about the current tools for research data sharing and the current
status of research data management. The goal of this user centric approach was
to formulate a system that would not be just another information system that
nobody uses but a system that would provide value for both the users that
put their research data to the system and the users that would get research
data out of the system.

\iffalse
This is a chapter where we rigorously define the problem statement.

Here is the problem background:

\begin{itemize}
    \item there is a growing demand on making research data public for other
          scientist and the public
    \item demand from funding agencies and open access community
    \item also scientific principles - how are you going to replicate others'
          work without the same datasets?
    \item furthering science
    \item these points need to be made concise and be backed up with literature
\end{itemize}

And here is what practical problems it will pose:

\begin{itemize}
    \item no infrastructure
    \item no know-how
    \item no culture
    \item there is demand, but there are no repercussions for not sharing the
          data
    \item many datasets are not suitable for publishing as is
\end{itemize}

\fi
