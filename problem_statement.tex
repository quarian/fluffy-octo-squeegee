\chapter{Problem Statement}
\label{chapter:problem}

\section{Research questions}

The primary research questions of this thesis are how research data can be shared
and published
using modern tools and how these tools work in practice. The rationale behind these questions is the increased
value of research data. Advances in sharing technologies and the requirements
for research data publication from the funding bodies and the research community
make research data sharing and publishing is a current problem.

During the research it became clear that there are other non technical
factors that affect the sharing and publishing of research data within the scientific community. In light
of this the secondary research question of the thesis is what non technical matters
affect the sharing and publishing of research data.

The non technical matters
also concern research data management during the research process, which means
that tools for research data sharing and publishing can not be examined without
some examination of tools and practices for research data management. The
thesis does not focus on research data management tools, but covers some of
them and looks into the research data publication and sharing tools also from
the angle of research data management during the research process.

\section{The contributions of this thesis}

The main contribution of this thesis is the technical examination of existing
software solutions to research data sharing and publishing. The most widespread tools are
benchmarked and from among them one solution, Harvard Dataverse, is examined
in further detail for deeper understanding of the functioning of these systems.
The existing research data publication platforms are fairly similar, which
makes examination like this applicable to the other solutions as well.

In addition to the technical examination this thesis proposes a set of
requirements that could be used as a basis of designing a research data
management and publication system or validate an existing system.
The requirements are derived from the background research and the user tests
conducted.

This thesis also provides an overview of the cultural atmosphere that surrounds
research data management and publishing. In light of the learnigs presented in
the thesis survival strategies regarding research data management, publishing and
sharing for research institutions in the new world of
data intensive science are proposed and ranked.

\section{The research approach}

To address the research problems the following methods are used:

\begin{itemize}
    \item Literature review,
    \item Expert interviews,
    \item Questionnaires,
    \item Tehcnical benchmarks and tests; and
    \item User tests.
\end{itemize}

Literature review is used to gain background knowledge of research data
management and publishing. Expert interviews are used to position the thesis
in the Finnish field - there is no point in working on something similar that
is already being solved. The expert interviews also aim to see if the findings
from the literature apply in practice.

Aalto University has is currently forming a policy about research data
management and as a part of that project two questionnaires about the
current needs of research data management have been conducted. The results of
those questionnaires are presented as additional evidence for the needs and
current status of research data management.

Existing technical solutions are benchmarked and examined in order to
understand the current options available. In order to facilitate user tests
one solution is going to be selected to act as a tool to conduct the user
tests.

Since the questionnaires already exist to shed light on the current needs user
tests were chosen as the method to extract most value towards designing a
system for research data management and publishing. The reasoning is that since
a system that relies this heavily on the users, in this case mostly the
research scientists, the users should be heard first and foremost. Contextual
interviews and tests with lead users were conducted to gain deeper
understanding about the current tools for research data sharing and the current
status of research data management. The goal of this user centric approach was
to formulate a system that would not be just another information system that
nobody uses but a system that would provide value for both the users that
put their research data to the system and the users that would get research
data out of the system.

\iffalse
This is a chapter where we rigorously define the problem statement.

Here is the problem background:

\begin{itemize}
    \item there is a growing demand on making research data public for other
          scientist and the public
    \item demand from funding agencies and open access community
    \item also scientific principles - how are you going to replicate others'
          work without the same datasets?
    \item furthering science
    \item these points need to be made concise and be backed up with literature
\end{itemize}

And here is what practical problems it will pose:

\begin{itemize}
    \item no infrastructure
    \item no know-how
    \item no culture
    \item there is demand, but there are no repercussions for not sharing the
          data
    \item many datasets are not suitable for publishing as is
\end{itemize}

\fi
