\chapter{Problem Statement}
\label{chapter:problem}

The primary research question of this thesis is how research data can be shared
using modern tools? The rationale behind the question is that due to the
value of research data, advance of sharing technologies and the requirements
to make research data public publishing research data is a current problem.

During the research it became clear that there are other than technical
problems that prevent sharing data within the scientific community. In light
of this this thesis also answers the question of what is preventing researchers
from publishing their research data?

The experimental part of the thesis focuses on the first question, but it is
clear that without answering and providing solutions to the second question
sharing research data is not going to happen. This thesis proposes some ways
to start the work of removing the barriers for reserch data sharing.

\iffalse
This is a chapter where we rigoriously define the problem statement.

Here is the problem background:

\begin{itemize}
    \item there is a growing demand on making research data public for other
          scientist and the public
    \item demand from funding agencies and open access community
    \item also scientific principles - how are you going to replicate others'
          work without the same datasets?
    \item furthering science
    \item these points need to be made concise and be backed up with literature
\end{itemize}

And here is what practical problems it will pose:

\begin{itemize}
    \item no infrastructure
    \item no know-how
    \item no culture
    \item there is demand, but there are no repercussions for not sharing the
          data
    \item many datasets are not suitable for publishing as is
\end{itemize}

\fi
